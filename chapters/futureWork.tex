\chapter{Μελλοντικές Επεκτάσεις}
\label{futureWork}
Το θέμα της παρούσας διπλωματικής εργασίας επιδέχεται πολλών μελλοντικών επεκτάσεων, αλλά και
διαφορετικών προσεγγίσεων. Περαιτέρω τροποποιήσεις από αυτές που περιγράφονται σε αυτό το κεφάλαιο μπορεί να βρεί κανείς στο \cite{allamanis11}. Παρακάτω παραθέτουμε μερικές από τις τροποποιήσεις και επεκτάσεις του αλγορίθμου εκπαίδευσης που η ανάλυση του GMl-ASLCS μας έκανε να θεωρήσουμε ως σημαντικές.

\section{Μη επίμονη ενημέρωση παραμέτρων των κανόνων του ΜαΣΤ}
Μετά το σχηματισμό του Match Set για ένα δεδομένο δείγμα εκπαίδευσης, ακολουθεί η ενημέρωση των παραμέτρων του κάθε κανόνα που συμμετέχει σε αυτό. Αυτή η διαδικασία είναι διαρκής: οι παράμετροι ενός κανόνα ανανεώνονται αέναα για κάθε Match Set στο οποίο συμμετέχει, μέχρι το τέλος της διαδικασίας εκπαίδευσης. Θα ήταν, ίσως, άσκοπο, και σίγουρα περισσότερο χρονοβόρο, να ενημερώνουμε τις παραμέτρους που δεν άπτονται, άμεσα ή έμμεσα, της αντικειμενικής αξιολόγησης του κανόνα, για το διάστημα στο οποίο του παρουσιάζεται το σύνολο δεδομένων $D$ για πολλοστή φορά. 

Πιο συγκεκριμένα, το σύστημα αποκτά μία μερική εικόνα της ποιότητας του κανόνα, μέσω των μεταβλητών
$true \:positive (tp)$, $match \:set \: appearances (msa)$, $accuracy = tp / msa$ και $fitness = f(tp/msa)$, αφού αυτός αξιολογηθεί για κάθε δείγμα που καλύπτει. Παρακάτω εξετάζουμε την κάθε μία.


\begin{itemize}

\item $tp$: Για κάθε \emph{label} $l$ για το οποίο συμφωνεί ο κανόνας με το δείγμα, η μεταβλητή $tp$ αυξάνεται ισόποσα με το $msa$, στην περίπτωσή μας, κατά ένα. Αν ο κανόνας αδιαφορεί για την $l$, τιμωρείται και το μέγεθος $tp$ αυξάνεται κατά μία θετική ποσότητα, πάντα μικρότερη από αυτήν με την οποία αυξάνεται εάν συμφωνεί. Στην περίπτωση που διαφωνεί, μένει ως έχει. Όταν ο κανόνας εξετάσει όλα τα δείγματα που μπορεί να καλύψει από το $D$ για πρώτη φορά, η τιμή της μεταβλητής θα ανήκει στο διάστημα $\left[0, coveredInstances \cdot labels\right]$. Από εκεί και έπειτα, η τιμή αυτή θα \emph{N}-πλασιάζεται στο τέλος της \emph{N}-ιοστής φοράς που ο κανόνας εξετάζει όλα τα δείγματα που καλύπτει. Όλη η χρήσιμη πληροφορία, όμως, έχει εξαχθεί ήδη από το τέλος της πρώτης φοράς που του παρουσιάζεται το $D$.
  
\item $msa$: Το $msa$ αυξάνεται \emph{ακριβώς} κατά ένα σε κάθε περίπτωση. Αντίστοιχα με το $tp$, η τιμή του \emph{N}-πλασιάζεται στο τέλος της \emph{N}-ιοστής φοράς που ο κανόνας εξετάζει όλα τα δείγματα του $D$ που καλύπτει.

\item $accuracy$ και $fitness$: Την πρώτη φορά που ο κανόνας θα εξετάσει όλα τα δείγματα του $D$, θα αποκαλυφθεί η εικόνα της ποιότητας του κανόνα, μέσω του αντικειμενικού προσδιορισμού της ακρίβειας και της καταλληλότητάς του. Η τιμή των δύο μεταβλητών θα βρίσκεται στο διάστημα $\left[0,1\right]$. Με εξαίρεση τους κανόνες που διαθέτουν $accuracy = fitness = 1$ ή $0$, η τιμή αυτών των μεταβλητών κατά την ενημέρωση των παραμέτρων τους σε μία τυχαία επανάληψη θα ταλαντώνεται γύρω από την πραγματική τους τιμή: αυτή που απέκτησαν στο τέλος της πρώτης φοράς που ο κανόνας εξέτασε όλα τα δείγματα του συνόλου δεδομένων. 

Θεωρητικά, υπάρχουν περιπτώσεις στις οποίες η τοποθέτηση των δειγμάτων μέσα στο $D$, δηλαδή η σειρά με την οποία παρουσιάζονται τα δείγματά του στο ΜαΣΤ, μπορεί να αναπτύξει συνθήκες όπου ένας κανόνας υποσκάπτει έναν άλλο (για παράδειγμα στην επιλογή για γονέα στο γενετικό αλγόριθμο) σε ένα χρονικό σημείο όπου δύο κανόνες έχουν λανθασμένη εκτίμηση για την καταλληλότητά τους, με τον έναν να την υπερεκτιμά και τον άλλον να την υποεκτιμά. Ίσως, από την άλλη, αυτό προσδίδει μία ευελιξία στην επιλογή γονέων για το γενετικό αλγόριθμο αυξάνοντας την ποικιλομορφία των παραγόμενων κανόνων. Ίσως ακόμα έχει θετικές δράσεις σε κομμάτια του συστήματος που δεν μπορούμε να προβλέψουμε λόγω της στοχαστικής του φύσης. Σε κάθε περίπτωση όμως, θεωρούμε ότι η αμεταβλητότητα αυτών των μεγεθών πέραν της πρώτης παρουσίασής του συνόλου δεδομένων στο ΜαΣΤ, παρέχουν μία αντικειμενική γνώμη στο σύστημα για τους κανόνες που εξελίσσει.

\end{itemize}

Συνολικά, θεωρούμε ότι η πειραματική αξιολόγηση της ανανέωσης των παραμέτρων $tp$, $msa$ και $fitness=f(tp, msa)$ των κανόνων ενός ΜαΣΤ μόνο για τα πρώτα $\abs{D}$ δείγματα που παρουσιάζονται σε αυτόν μετά τη δημιουργία του, θα άξιζε λόγω της μείωσης του χρόνου εκπαίδευσης και της “ορθολογικότητας" που εισάγει. Επιπλέον, αυτός ο τρόπος ανανέωσης είναι απαραίτητος για την επέκταση που προτείνουμε στην επόμενη παράγραφο.


\section{Αντικατάσταση της μεθόδου υπολογισμού της Ακρίβειας των Κανόνων}
Η ακρίβεια των κανόνων, όπως έχουμε δει, βασίζεται στην αύξηση των μεγεθών $tp$ και $msa$ κατά τις ποσότητες $\omega$ και $\phi$ αντίστοιχα, για κάθε δείγμα που καλύπτουν και κάθε ετικέτα $l$. 

\begin{equation}
accuracy(t+1) = \dfrac{tp(t) + \omega}{msa(t) + \phi}
\end{equation}
\\

Μία τολμηρή κίνηση, που ξεφεύγει από το ακριβειο-κεντρικό πλαίσιο λειτουργίας των ΜαΣΤ, είναι η αντικατάσταση της ακρίβειας, όπως αυτή υπολογίζεται με τον παραπάνω τρόπο, με ένα μέγεθος που αντικαθιστά την πρόσθεση με τον πολλαπλασιασμό. Για αντίστοιχες μεταβλητές $\omega'$ και $\phi'$, η καταλληλότητα ενός κανόνα μπορεί τότε να πάρει τη μορφή

\begin{equation}
fitness(t+1)  = \left(\dfrac{tp(t) \cdot \omega'}{msa(t) \cdot \phi'}\right)^{\nu}
\end{equation}
\\

Όπως αναφέραμε στην προηγούμενη παράγραφο, αυτή η μέθοδος υπολογισμού είναι σημαντικό να χρησιμοποιείται με ενημέρωση των παραμέτρων $tp$ και $msa$ μόνο κατά την πρώτη φορά που παρουσιάζεται το σύνολο δεδομένων $D$ σε έναν κανόνα, λόγω της βιαιότητας με την οποία μπορεί να επιδράσει η πράξη του πολλαπλασιασμού στις τιμές της καταλληλότητας των κανόνων. 

Σε αυτόν τον τρόπο, η ρύθμιση των $(\omega', \phi')$ θα πρέπει να είναι πολύ περισσότερο προσεκτική από αυτή των $(\omega, \phi)$ και η τιμή της παραμέτρου $\omega'$ σίγουρα πολύ υψηλότερη από την τιμή $0.9$, ίσως στα επίπεδα $\omega'=0.99$ για $\phi'=1$. Από αυτή τη μέθοδο δεν είναι σίγουρο ότι μπορούμε να περιμένουμε κάτι συγκλονιστικό, αν και σε μερικά σύνολα δεδομένων, όπως το τεχνητό $mlPosition_{7}$ έχει δείξει ότι το ΜαΣΤ συγκλίνει γρηγορότερα και καλύτερα από την ακριβειοκεντρική προσέγγιση του GMl-ASLCS. 

Σε κάθε περίπτωση, η προσπάθεια για ριζοσπαστική σκέψη για τροποποίηση, ακόμα και των πιο βασικών τμημάτων ενός ΜαΣΤ, αξίζει γιατί μπορούν να εξαχθούν ενδιαφέροντα συμπεράσματα, όχι μόνο για την ίδια τη λειτουργία της μεθόδου τροποποίησης, αλλά και για την εν γένει συμπεριφορά ενός ΜαΣΤ.


\section{Περί του $\theta_{GA}$}
Το κατώφλι εμπειρίας $\theta_{GA}$, όπως έχουμε ξαναναφέρει, υποδεικνύει το ρυθμό με τον οποίο παράγονται οι κανόνες απόγονοι μέσω του Γενετικού Αλγορίθμου. Πέραν αυτού, για δεδομένο συνδυασμό $$(\abs{I}, \theta_{GA}, maxPopulationSize, P_{\#A})$$ όπου $\abs{I}$ ο αριθμός επαναλήψεων μάθησης, $maxPopulationSize$ το μέγιστο πλήθος των μικρο-κανόνων του πληθυσμού και $P_{\#A}$ η πιθανότητα γενίκευσης γνωρισμάτων των κανόνων που παράγονται από το τμήμα Κάλυψης, το κατώφλι $\theta_{GA}$ ρυθμίζει έμμεσα το βαθμό γενίκευσης των κανόνων, τόσο μέχρι ο αριθμός τους να φτάσει το μέγιστο $maxPopulationSize$, όσο και από εκεί και έπειτα, μέχρι το τέλος των επαναλήψεων εκπαίδευσης.

Όλα τα ΜαΣΤ, από όσο γνωρίζουμε, χρησιμοποιούν μία και μοναδική σταθερή τιμή για αυτή τη μεταβλητή. Στην προσπάθειά μας να ρυθμίσουμε με ακριβέστερο τρόπο το βαθμό γενίκευσης των κανόνων ενός ΜαΣΤ, θα μπορούσαμε να χρησιμοποιήσουμε αντί για μία τιμή, δύο. Μία τιμή $\theta_{GA_{0}}$ μέχρι ο αριθμός των κανόνων να φτάσει το μέγιστο αριθμό τους, ώστε να οδηγήσουμε τον πληθυσμό στο επιθυμητό επίπεδο μέσης κάλυψης δειγμάτων, το οποίο θα είναι και το μέγιστό και μία δεύτερη $\theta_{GA_{1}}$, εν γένει διαφορετική από την πρώτη, ώστε στο τέλος της εκπαιδευτικής διαδικασίας ο πληθυσμός να έχει φτάσει στο τελικό επίπεδο μέσης κάλυψης που έχουμε θέσει ως στόχο.


\begin{equation}
\theta_{GA} = \left\{
\begin{array}{ c l }
	\displaystyle \theta_{GA_{0}}, & rouletteWheelDeletionsCommenced = false
	\\
	\displaystyle \theta_{GA_{1}}, & rouletteWheelDeletionsCommenced = true

\end{array}
\right.
\end{equation}
\\

Με αυτό τον τρόπο, ο σχεδιαστής αποκτά μεγαλύτερη ευελιξία στο έργο του, καθώς μπορεί να ρυθμίσει το χρονικό σημείο στο οποίο η μέση κάλυψη των κανόνων θα αρχίσει να φθίνει (το σημείο στο οποίο ο πληθυσμός αποκτά τη μέγιστή του χωρητικότητα) και τον αριθμό των διαγραφών που θα συμβούν, λόγω της πρόσθεσης κανόνων στον πληθυσμό, ανεξάρτητα το ένα από το άλλο.


\section{Περί του $\theta_{exp}$}
Όπως αναφέραμε στην Παρ. \ref{subsec:altFitnessInExp}, ίσως είναι σκόπιμο να εγκαταλείψουμε την έκπτωση της καταλληλότητας των κανόνων βάσει της εμπειρίας τους στη συνιστώσα Εξερεύνησης, λόγω της δυσκολίας στον προσδιορισμό μίας βέλτιστης τιμής για κάθε σύνολο δεδομένων, αλλά και ώστε κάθε κανόνας, ανεξάρτητα από τον αριθμό των δειγμάτων που καλύπτει, να αντιμετωπίζεται ισότιμα σε σχέση με τους υπόλοιπους, όσον αφορά στο από ποιο χρονικό σημείο και έπειτα μπορεί να συμμετέχει στην εξελικτική διαδικασία. Το αντικειμενικό κριτήριο για κάθε κανόνα $i$ θα μπορούσε να είναι το χρονικό σημείο στο οποίο ο κανόνας έχει αξιολογηθεί πάνω σε όλα τα δείγματα του συνόλου δεδομένων $D$ που αυτός μπορεί να καλύψει, δηλαδή το σημείο στο οποίο για πρώτη φορά έχει παρουσιαστεί στον κανόνα το $D$ στο σύνολό του.


\begin{equation}
fitness(i) = \left\{
\begin{array}{ c l }
\displaystyle 0, 					& instancesChecked < \abs{D}
\\
\displaystyle (accuracy(i))^{\nu}, 	& $αλλού$
\end{array}
\right.
\end{equation}
\\


Βέβαια, με αυτό τον τρόπο θα περιμέναμε μία χαμηλότερη πίεση προς γενίκευση, γιατί πλέον καθίσταται άσχετος ο αριθμός δειγμάτων που καλύπτει ένας κανόνας και άρα το πότε αυτός καταφέρνει να υπερπηδήσει το κατώφλι $\theta_{exp}$.


\section{Περί των διαγραφών}
Στον GMl-ASLCS υπάρχουν δύο συνιστώσες διαγραφών: η πλέον καθιερωμένη διαγραφή με επιλογή κανόνων από τον πληθυσμό (εδώ με επιλογή ρουλέτας) και η διαγραφή κανόνων με κριτήρια διαγραφής πάνω στα σύνολα Match Set. Η λογική και των διαγραφών μπορεί να επεκταθεί περαιτέρω. Ίσως είναι σκόπιμο να προσπαθήσουμε να διαγράφουμε, υπό συνθήκες, κανόνες με τον πρώτο τρόπο, όχι μόνο από τον πληθυσμό, αλλά και από τα σύνολα Match Set $[M]$ ή από τα Incorrect Set $[!C]$, λόγω του μικρότερου αριθμού τους, αλλά και των αντικειμενικών συνθηκών κάτω από τις οποίες σχηματίζονται τα παραπάνω σύνολα. Αντίστοιχα, η δεύτερη μέθοδος διαγραφής, όπως προαναφέραμε στην Παρ. \ref{subsec:controlInM}, ίσως χρειάζεται να χρησιμοποιείται σε διαφορετικά σύνολα εκτός του Match Set.

Επιπρόσθετα, ανεξάρτητα από την προσπάθειά μας για αύξηση του μέσου αριθμού δειγμάτων που καλύπτουν οι κανόνες ενός ΜαΣΤ, η λογική της διαγραφής κανόνων από τα Match Sets μπορεί να επεκταθεί και σε περισσότερα επίπεδα κάλυψης εκτός από το χαμηλότερο. Η διαγραφή κανόνων σε \emph{κάθε} επίπεδο κάλυψης ίσως αποτελέσει κάτι το βοηθητικό προς την εκπαιδευτική διαδικασία, διαγράφοντας κανόνες με χαμηλή καταλληλότητα εν γένει. Στην παραπάνω περίπτωση, ίσως να ήταν χρήσιμο να θεωρήσουμε ένα επιπλέον ανώφλι καταλληλότητας (ή και κάποιο άλλο κριτήριο) λόγω της μεγάλης αλλαγής που η παραπάνω λειτουργία θα επιφέρει στην εξελικτική διαδικασία.


\section{Συσχετίσεις Ετικετών}
Από την ανάλυσή μας μέχρι στιγμής απουσιάζει μία μεθοδολογία που να λαμβάνει υπόψη τη συσχέτιση των ετικετών ενός πολυκατηγορικού προβλήματος για τον ακριβέστερο συμπερασμό των ετικετών αταξινόμητων δειγμάτων. Στην Παρ. \ref{subsect:labelAssociation} είδαμε τους δύο τρόπους με τους οποίους μπορούμε οπτικά να αναγνωρίσουμε το βαθμό αλληλοσυσχετίσεων των ετικετών ενός προβλήματος, αλλά η συνολική μεθοδολογία μας δεν χρησιμοποιεί κάποιον από αυτούς. Από τους δύο τρόπους, οι Χάρτες Θερμότητας θα ήταν περισσότερο πρόσφοροι για την εφαρμογή τους στο συνολικό πλαίσιο των πολυκατηγορικών ΜαΣΤ, λόγω της άμεσης μετάφρασής τους σε πίνακες πιθανοτήτων. 

Πιθανές μεθοδολογίες θα μπορούσαν να χρησιμοποιούν τις παραπάνω πιθανότητες για την εφαρμογή κάποιου είδους επιπρόσθετης έκπτωσης στην καταλληλότητα των κανόνων κατά την εξελικτική διαδικασία, ή ακόμα την τροποποίηση του τμήματος Κάλυψης ώστε να υπάρχει κάποια “πρόταση" ετικετών σε κανόνες. Επιπρόσθετα, εκτός από τη ρύθμιση του κατωφλίου στη συνιστώσα Επίδοσης, το ΜαΣΤ θα μπορούσε να επιλέξει ένα μικρότερο υποσύνολο των κανόνων που θα αποφασίσουν για την κατηγοριοποίηση ενός αταξινόμητου δείγματος με βάση το πόσο πιθανό είναι να κατηγοριοποιηθεί σε μία συγκεκριμένη ετικέτα.



\section{Περί της εκτίμησης του μεγέθους $cs$}
Όπως είδαμε στην Παρ. \ref{subsec:gmlaslcsDeletion}, ο GMl-ASLCS χρησιμοποιεί την εκτίμηση του μέσου ελαχίστου μεγέθους των Correct Sets στα οποία συμμετέχει ένας κανόνας, αντί για την ίδια την τιμή του, για όλα τα Correct Sets στα οποία συμμετέχει. Αυτή η μεθοδολογία είδαμε πώς επηρεάζει την αντιμετώπιση των κανόνων του ΜαΣΤ, μέσω της υπερ-εκτίμησης και υπο-εκτίμησης της πραγματικής τιμής του $cs$. Για τον υπολογισμό του πραγματικού $cs$ ενός κανόνα $rule$, αντί της Εξ. \ref{eq:csGMlASLCS} που χρησιμοποιεί το ρυθμό μάθησης $\beta$, μπορεί να χρησιμοποιηθεί ο απευθείας υπολογισμός του $cs$ μέσω της

\begin{equation}
rule.cs(t) = \frac{(t-1) \cdot rule.cs(t-1) + cs(t)}{t}
\end{equation}
\\
όπου $t$ και $t-1$ είναι o αριθμός των φορών που έχει συμμετέχει ο $rule$ σε Correct Set, $cs(t)$ το μέγεθος του Correct Set στο οποίο συμμετέχει ο $rule$ την $t$ φορά και $rule.cs(t)$ η μέση τιμή του $cs$ του $rule$ την τρέχουσα ($t$) στιγμή.
 

\section{Διαμοιρασμός Καταλληλότητας}
Ο διαμοιρασμός καταλληλότητας (fitness sharing) που χρησιμοποιεί ο UCS στοχεύει στο να υπάρχει ποικιλομορφία στον πληθυσμό των κανόνων που εξελίσσει, τιμωρώντας τους υπεργενικούς κανόνες που συμμετέχουν στα ίδια Correct Sets με κανόνες που είναι καταλληλότεροι από αυτούς. Αυτό παράγει μία επιπρόσθετη πίεση προς διαγραφή υπεργενικών κανόνων και μία επιλεκτική πίεση προς ανακάλυψη ειδικών κανόνων \cite{orriols2008revisiting}. 

Ο GMl-ASLCS θα μπορούσε να χρησιμοποιήσει τις παραπάνω ευεργετικές ιδιότητες του διαμοιρασμού καταλληλότητας σε συνδυασμό με τις ήδη υπάρχουσες λειτουργίες του, ώστε να βελτιώσει την ποιότητα των γενικών του κανόνων, αλλά και να ανακαλύψει περισσότερους ειδικούς κανόνες που ίσως δεν καταφέρνει να βρει. Αν και ο διαμοιρασμός καταλληλότητας δεν έχει υλοποιηθεί, υποπτευόμαστε ότι στον πολυκατηγορικό χώρο θα αυξήσει την υπολογιστική πολυπλοκότητα του GMl-ASLCS ανάλογα με τον αριθμό των ετικετών του προβλήματος και το μέγεθος του πληθυσμού, αν διαμοιράσουμε και εδώ την καταλληλότητα των κανόνων των Correct Sets, δηλαδή ανά ετικέτα. Ένας εναλλακτικός τρόπος θα ήταν, αντί για το διαμοιρασμό της καταλληλότητας για κάθε ετικέτα, να συγκεντρώσουμε όλους τους κανόνες των επιμέρους Correct Sets σε ένα σύνολο, δηλαδή όλους τους κανόνες που συμφωνούν με τις επιμέρους ετικέτες ενός δείγματος, και να διαμοιράσουμε την καταλληλότητά τους σε αυτό το σύνολο.

