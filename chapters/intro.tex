\chapter{Εισαγωγή}
Η κατηγοριοποίηση και οργάνωση διαφορετικών οντοτήτων, αν και υπήρχε από την εποχή του Αριστοτέλη, τον τελευταίο αιώνα έχει γίνει ανάγκη της καθημερινότητας και ένα εργαλείο της. Η ραγδαία σύγκλιση των επιστημονικών εξελίξεων στις περιοχές της πληροφορικής και των επικοινωνιών, όμως, στις τελευταίες δεκαετίες, έχει οδηγήσει στην κατακόρυφη αύξηση των διαθέσιμων και εύκολα προσβάσιμων δεδομένων, λόγω και της μείωσης του κόστους αποθήκευσής τους. Αυτό είχε ως αποτέλεσμα τη συσσώρευση δεδομένων που, λόγου του όγκου τους, δεν είναι πλέον εύκολα διαχειρίσιμα από τους ανθρώπους. Παράλληλα, η αύξηση της επεξεργαστικής ισχύος των υπολογιστών οδήγησε στην ανάπτυξη προσεγγιστικών, αλλά ταυτόχρονα μεγάλης ακρίβειας, τρόπων επίλυσης προβλημάτων από αυτούς, μεταφέροντας το βάρος από τον Άνθρωπο στη Μηχανή.

Το έργο της κατηγοριοποίησης, συνεπώς, ήταν φυσιολογικό να γίνει αντικείμενο ενδελεχούς έρευνας, δίνοντας ώθηση για την ανάπτυξη του κλάδου της Μηχανικής Μάθησης, τομέας του οποίου είναι η Εξόρυξη Δεδομένων. Η Εξόρυξη Δεδομένων, έχει ως αντικείμενο την εξαγωγή έμμεσης, προηγουμένως άγνωστης και ενδεχομένως χρήσιμης πληροφορίας από δεδομένα, με σκοπό τη δημιουργία υπολογιστικών προγραμμάτων τα οποία διατρέχουν αυτόματα τα περιεχόμενα κάθε είδους βάσεων δεδομένων, αναζητώντας κανονικότητες ή πρότυπα. Τα πρότυπα αυτά, εφόσον ανακαλυφθούν, είναι δυνατόν να γενικευτούν, ώστε να χρησιμοποιηθούν για την πρόβλεψη μελλοντικών καταστάσεων. Η εξαγωγή ισχυρών και γενικεύσιμων προτύπων είναι μία αναπόφευκτα ανακριβής διαδικασία: τα ανακαλυπτόμενα πρότυπα μπορεί να είναι κοινότυπα ή άνευ ενδιαφέροντος, είτε να εξαρτώνται από τυχαίες συμπτώσεις, εγγενείς στα χρησιμοποιούμενα σύνολα δεδομένων. Επιπλέον, εφόσον τα πραγματικά δεδομένα είναι από τη φύση τους ατελή, περιέχοντας συχνά τμήματα που είναι αλλοιωμένα ή ελλειπή, οποιαδήποτε κανονικότητα ανακαλύπτεται σε αυτά έχει περιορισμένη ακρίβεια. Σε κάθε περίπτωση, η Μηχανική Μάθηση έχει ως απώτερο στόχο την επίλυση προβλημάτων που είναι σε τέτοιο βαθμό σύνθετα ώστε δεν είναι επιλύσιμα από τον Άνθρωπο ή, τουλάχιστον την παροχή οδηγιών για την επίλυσή τους \cite{weka}. 

Η αναζήτηση και προσέγγιση του βέλτιστου μοντέλου που αναπτύσσουν οι αλγόριθμοι μηχανικής μάθησης είναι η βασική προϋπόθεση κάθε μορφής υπολογιστικής νοημοσύνης και, συνεπώς, για την αποτελεσματικότερη επίλυση προβλημάτων, η αναζήτηση θα πρέπει να είναι καθολική και να μην περιορίζεται σε τοπικά βέλτιστες λύσεις. Ταυτόχρονα, μία καθολική αναζήτηση όλων των πιθανών καταστάσεων και λύσεων ενδέχεται να είναι απαγορευτική από άποψη χρόνου, ανάλογα με το μέγεθος του προβλήματος.

Επειδή το πρόβλημα της καθολικής αναζήτησης, λοιπόν, είναι δύσκολο και, επιπρόσθετα, σπάνια υπάρχουν σαφείς μαθηματικές λύσεις στα πραγματικά προβλήματα, οι μηχανικοί και οι επιστήμονες της πληροφορικής στράφηκαν προς τη Φύση για έμπνευση, ακολουθώντας την προσέγγιση της μίμησης της εξέλιξης των ειδών. Τα βιολογικά όντα δρουν στο (και σε σχέση με το) περιβάλλον τους, δοκιμαζόμενα σε διάφορες συνθήκες, ενώ τα γονίδιά τους διασταυρώνονται και μεταλλάσσονται αδιάκοπα στο πέρασμα του χρόνου. Παράλληλα, τα γονίδια που καθιστούν τα επιμέρους άτομα ενός είδους κατάλληλα για επιβίωση στο περιβάλλον τους διατηρούνται και κληροδοτούνται στις μελλοντικές γενιές, με μεγαλύτερη πίεση σε σχέση με αυτά που καθιστούν την επιβίωση (με την ευρεία έννοια) δυσχερέστερη, ακριβώς λόγω των μεγαλύτερων δυσκολιών των ατόμων που τα κατέχουν να επιβιώσουν και, συνεπώς, να αναπαραχθούν και να τα μεταλαμπαδεύσουν στους απογόνους τους. Το παραπάνω τμήμα της θεωρίας της εξέλιξης των ειδών που πρότεινε ο Δαρβίνος αποτελεί αναπόσπαστο κομμάτι του πλαισίου λειτουργίας των Γενετικών Αλγορίθμων (Genetic Algorithms).

Όπως και η θεωρία της εξέλιξης των ειδών, έτσι και οι Γενετικοί Αλγόριθμοι θεωρούν έναν πληθυσμό από άτομα-μοντέλα, η καταλληλότητα των οποίων εξαρτάται από την ικανότητα επιβίωσής τους στο περιβάλλον ενός προβλήματος, σε αντιστοιχία με το περιβάλλον των βιολογικών ειδών. Στους Γενετικούς Αλγορίθμους, τα μοντέλα αυτά τοποθετούνται στο εικονικό περιβάλλον-οικοσύστημα του προβλήματος, όπου εξελίσσονται με τον ίδιο τρόπο που εξελίσσονται οι οργανισμοί στη φύση. Ένα τέτοιο περιβάλλον είναι τα Μανθάνοντα Συστήματα Ταξινομητών (ΜαΣΤ), τα οποία δημιουργούν έναν πληθυσμό από κανόνες ταξινόμησης (άτομα), που εξελίσσονται μέσα σε ένα δεδομένο πλαίσιο μάθησης, με βάση τα χαρακτηριστικά του προβλήματος-στόχου προς επίλυση. Τα ΜαΣΤ χρησιμοποιούν κανόνες ταξινόμησης για την παραγωγή εύληπτων και κατανοητών λύσεων, από ειδικούς ή/και υπεύθυνους λήψης αποφάσεων, ακόμα και χωρίς κάποια εκπαίδευση στους υπολογιστές.

Βασισμένη στα παραπάνω, η παρούσα εργασία ασχολείται με ζητήματα Εξόρυξης Δεδομένων και το συνδυασμό μεθόδων Μηχανικής Μάθησης για τη δημιουργία προβλεπτικών μοντέλων. Πιο συγκεκριμένα, χρησιμοποιεί Μανθάνοντα Συστήματα Ταξινομητών, τα οποία ενσωματώνουν τους Γενετικούς Αλγορίθμους, για την καθολική αναζήτηση κανόνων ταξινόμησης δειγμάτων με βάση σύνολα δεδομένων και την κατηγοριοποίηση δεδομένων (δηλαδή των δειγμάτων) σε περισσότερες από μία κατηγορίες, οι οποίες ονομάζονται ετικέτες, ώστε να υλοποιήσει πολυκατηγορική ταξινόμηση.

Το φάσμα των εφαρμογών της πολυκατηγορικής ταξινόμησης σε πραγματικά προβλήματα είναι ευρύ, και αυτές κυμαίνονται από την κατηγοριοποίηση εγγράφων και πολυμεσικών οντοτήτων έως τη συσχέτιση βιολογικών λειτουργιών σε γονίδια και τη διάγνωση ιατρικών προβλημάτων.

Η χρήση ΜαΣΤ τύπου Michigan για την επίλυση πολυκατηγορικών προβλημάτων επιλέχθηκε λόγω των πλεονεκτημάτων τους έναντι των εναλλακτικών μεθόδων ταξινόμησης. Τα ΜαΣΤ δημιουργούν και μεταχειρίζονται κανόνες, των οποίων η μορφή είναι αντιληπτή και κατανοητή από τον άνθρωπο, αλλά και η ικανότητα γενίκευσής τους οδηγεί σε συμπαγείς χαρτογραφήσεις του χώρου γνωρισμάτων σε σχέση με το χώρο των κατηγοριών/ετικετών. Παράλληλα, τα ΜαΣΤ διαθέτουν μία ευέλικτη αναπαράσταση γνώσης, η οποία μπορεί να προσαρμοστεί εύκολα για την αντιμετώπιση νέων τύπων δεδομένων, ενώ κατασκευάζουν τα μοντέλα τους online, γεγονός που είναι κρίσιμης σημασίας για την επιτυχία τους σε προβλήματα με μεγάλο όγκο δεδομένων ή με δεδομένα που γίνονται σταδιακά διαθέσιμα, με μορφή ροών. Τέλος, τα ΜαΣΤ αποτελούν παραλληλοποιήσιμα συστήματα, με αποτέλεσμα να είναι εύκολη η επέκτασή τους σε μεγάλη κλίμακα, χωρίς ιδιαίτερα προβλήματα.


\section{Σκοπός της Διπλωματικής Εργασίας}
Η παρούσα διπλωματική εργασία καταπιάνεται με το πρόβλημα της πολυκατηγορικής ταξινόμησης με Μανθάνοντα Συστήματα Ταξινομητών (ΜαΣΤ) τύπου Michigan (Παρ. \ref{subsec:michiganPittsburg}). Πιο συγκεκριμένα, αναζητά τρόπους με τους οποίους μπορεί να επεκταθεί η υπάρχουσα υλοποίηση του GMl-ASLCS \cite{allamanis11}, του πρώτου αλγορίθμου που επιχειρεί την επίλυση πολυκατηγορικών προβλημάτων με χρήση ΜαΣΤ. Η επέκταση έχει ως απώτερο στόχο την αύξηση της προβλεπτικής ικανότητας αυτού του αλγορίθμου, μέσω της τροποποίησης των επιμέρους τμημάτων του και της επινόησης νέων λειτουργιών, που προσιδιάζουν στη φύση των ΜαΣΤ, αλλά και της πολυκατηγορικής ταξινόμησης. 

Στην εποπτευόμενη μάθηση, η ικανότητα ταξινόμησης ενός μοντέλου προσδιορίζεται από την ικανότητά του για ακριβή αντιστοίχιση ενός συνόλου κατηγοριών σε κάθε δείγμα ενός συνόλου δεδομένων, τα περιεχόμενα του οποίου είναι άγνωστα στο μοντέλο. Με άλλα λόγια πρόκειται για δείγματα με τα οποία το μοντέλο δεν έχει εκπαιδευτεί. Για να επιτύχει την αντιστοίχιση ένα ΜαΣΤ, θα πρέπει καθένας από τους κανόνες που χρησιμοποιεί να είναι σε θέση να ενεργοποιείται από ένα σύνολο άγνωστων προς αυτόν δειγμάτων και να τα ταξινομεί με ακρίβεια. Συνεπώς, το μοντέλο στο σύνολό του θα πρέπει να μπορεί να γενικεύει πάνω στο σύνολο δειγμάτων με το οποίο εκπαιδεύεται. Η παρούσα εργασία, λοιπόν, κινείται σε δύο άξονες: α) στοχεύει στην αύξηση του αριθμού των δειγμάτων που είναι ικανό να κατηγοριοποιήσει το μοντέλο που παράγει ο GMl-ASLCS, χωρίς να μειώνεται η προβλεπτική του ικανότητα, και β) στην αύξηση της ακρίβειας κατηγοριοποίησής του, αλλά και τη βελτίωση των υπολοίπων μετρικών που χρησιμοποιούνται για την αξιολόγηση των επιδόσεών του.



\section{Μεθοδολογία Εργασίας}
Η εκπόνηση της παρούσας εργασίας ξεκίνησε τον Απρίλιο του 2012 και ολοκληρώθηκε το Μάιο του 2013. Σε πρώτο στάδιο, στόχος υπήρξε η εξοικείωση με την ευρύτερη περιοχή του γνωστικού πεδίου της εργασίας. Για αυτό το σκοπό μελετήθηκε ένα σύνολο βιβλίων σχετικά με τον τομέα της Μηχανικής Μάθησης και των Γενετικών Αλγορίθμων. Η δεύτερη φάση περιελάμβανε την εξοικείωση με το ήδη υλοποιημένο λογισμικό, σε γλώσσα Java, την οργάνωση του κώδικά του και τις διάφορες λειτουργίες των τμημάτων που το απαρτίζουν. 

Παράλληλα, μελετήθηκε βιβλιογραφία γύρω από τα Μανθάνοντα Συστήματα Ταξινομητών για την περίπτωση της μονοκατηγορικής ταξινόμησης και η εργασία που κυοφόρησε την απαρχή των ΜαΣΤ στον πολυκατηγορικό χώρο. Για την περαιτέρω εμβάθυνση στις εσωτερικές διεργασίες του λογισμικού, μελετήθηκαν τα εσωτερικά του, ξεχωριστά, τμήματα, ως προς την αρχή και τον τρόπο λειτουργίας τους. 

Η επόμενη φάση περιελάμβανε την ανάγνωση επιστημονικών άρθρων σχετικά με την πολυκατηγορική ταξινόμηση που, λόγω της απουσίας βιβλιογραφίας της αντιμετώπισής της με ΜαΣΤ, κινούνταν στον αιτιοκρατικό χώρο. Εκεί ξεκίνησε και η μεγαλύτερη συγκέντρωση προς το λογισμικό. Στην πρώτη φάση της εργασίας, διενεργούνταν πειράματα πάνω στα τεχνητά σύνολα δεδομένων, λόγω του μικρού χρόνου εκτέλεσής τους, και στη συνέχεια ακολούθησαν πειράματα με τα πραγματικά σύνολα δεδομένων. Σε αυτό το στάδιο άρχισε και η εργασία της βελτιστοποίησης του GMl-ASLCS και της εύρεσης αποτελεσματικών διαφοροποιήσεων και καινούριων λειτουργιών. Κάθε προσθήκη ή τροποποίηση ελεγχόταν ως προς τη συνεισφορά της πάνω στο σύνολο των πραγματικών συνόλων δεδομένων και, στη συνέχεια, των τεχνητών. 

Αφού ολοκληρώθηκε η φάση της βελτιστοποίησης και συγκεντρώθηκαν στον τελικό GMl-ASLCS όλες οι τροποποιημένες και επιπρόσθετες λειτουργίες, ξεκίνησε το τελικό στάδιο εκπαίδευσης του GMl-ASLCS πάνω στα έξι πραγματικά σύνολα δεδομένων που εξετάστηκαν σε αυτή την εργασία. Στη συνέχεια, διενεργήθηκαν πειράματα πάνω σε τέσσερα τεχνητά σύνολα δεδομένων, δοκιμάζοντας τη συμπεριφορά του GMl-ASLCS, όσο και των βασικών αλλαγών που επιφέραμε σε αυτόν. Στο τελικό στάδιο, πραγματοποιήθηκε καταγραφή της ανάλυσης και των συμπερασμάτων που προέκυψαν κατά τη διάρκεια της εκπόνησης αυτής της εργασίας.

\section{Διάρθρωση της Αναφοράς}
Η παρούσα Διπλωματική Εργασία αποτελείται από τρία μέρη:

Στο \textbf{Μέρος Ι} περιγράφονται οι βασικές γνώσεις και μέθοδοι των γνωστικών πεδίων στα οποία στηρίχθηκε η εργασία. Πιο συγκεκριμένα, γίνεται αναφορά στη Μηχανική Μάθηση (Κεφάλαιο \ref{machineLearningAndClassification}) και αναλύονται οι βασικές έννοιες και προσεγγίσεις της Εξόρυξης Δεδομένων που έχουν αναπτυχθεί τόσο για την απλή, όσο και για την πολυκατηγορική ταξινόμηση. Στο Κεφάλαιο \ref{geneticAlgorithms} καταγράφονται οι βασικές έννοιες των Γενετικών Αλγορίθμων, όπως η έννοια του πληθυσμού, της καταλληλότητας και της φυσικής επιλογής, ενώ στο Κεφάλαιο \ref{learningClassifierSystems} περιγράφεται η δομή και η ιστορία των Μανθανόντων Συστημάτων Ταξινομητών (ΜαΣΤ). Πιο συγκεκριμένα, γίνεται αναφορά στα πρώτα ΜαΣΤ που χρησιμοποιούν Ενισχυτική Μάθηση και σε μεταγενέστερα που χρησιμοποιούνται σε προβλήματα Εποπτευόμενης Μάθησης.

Στο \textbf{Μέρος ΙΙ} καταγράφεται πλήρως ο τρόπος με τον οποίο επιχειρήθηκε να αντιμετωπιστεί το πρόβλημα της πολυκατηγορικής ταξινόμησης με Μανθάνοντα Συστήματα Ταξινομητών, έχοντας ως αφετηρία τον αλγόριθμο GMl-ASLCS, τον οποίο στο πλαίσιο της εργασίας ονοματοδοτούμε ως GMl-ASLCS$_{\:0}$. Αρχικά, στο Κεφάλαιο \ref{gmlaslcs0} παρουσιάζεται επιγραμματικά η δομή του GMl-ASLCS$_{\:0}$ και δίνεται έμφαση σε μερικές από τις λειτουργίες του, ώστε να γίνουν κατανοητοί οι λόγοι και η σημασία των αλλαγών που επιφέραμε σε αυτόν. Στο Κεφάλαιο \ref{gmlaslcs} παρουσιάζεται η δομή του GMl-ASLCS, δηλαδή του τροποποιημένου GMl-ASLCS$_{\:0}$, ο κύκλος εκπαίδευσής του και αναλύονται μερικές από τις λειτουργίες των επιμέρους τμημάτων του, οι οποίες τροποποιήθηκαν. Παράλληλα, προτείνονται μερικές περαιτέρω τροποποιήσεις. Στο Κεφάλαιο \ref{testBedsExperiments} αξιολογούνται οι επιδόσεις του GMl-ASLCS και η επίδραση των τεσσάρων βασικών αρθρωτών αλλαγών που επιφέραμε στη λειτουργία του GMl-ASLCS$_{\:0}$, σε τέσσερα τεχνητά σύνολα δεδομένων. Στο Κεφάλαιο \ref{realWorldDatasetsExps} αξιολογούνται οι επιδόσεις του GMl-ASLCS, αλλά και των προτεινόμενων τροποποιήσεών του σε έξι πραγματικά σύνολα πολυκατηγορικών δεδομένων.

Τέλος, στο \textbf{Μέρος ΙΙΙ} καταγράφονται τα συμπεράσματα που προέκυψαν από την παρούσα εργασία (Κεφάλαιο \ref{conclusions}), καθώς και οι πιθανές μελλοντικές τροποποιήσεις και επεκτάσεις του GMl-ASLCS (Κεφάλαιο \ref{futureWork}).