\chapter{Συμπεράσματα}
\label{conclusions}
Στην παρούσα διπλωματική εργασία παρουσιάστηκε η δεύτερη και βελτιστοποιημένη μορφή του Μανθάνοντος Συστήματος Πολυκατηγορικής Ταξινόμησης GMl-ASLCS. Αρχικά περιγράψαμε τα τμήματα τα οποία είναι κοινά ανάμεσα στον GMl-ASLCS και τον GMl-ASLCS$_{\:0}$ και αφορούν:

\begin{itemize}
\item στην αναπαράσταση του τμήματος συνθήκης και απόφασης των κανόνων που χρησιμοποιούν τα δύο ΜαΣΤ

\item στη βασισμένη-στην-εμπειρία έκπτωση της καταλληλότητας των κανόνων και τη λειτουργία του τμήματος Κάλυψης στη συνιστώσα Εξερεύνησης

\item στη μεθοδολογία υπολογισμού του μέσου μεγέθους των Correct Sets στα οποία συμμετέχουν οι κανόνες των ΜαΣΤ

\item στο Γενετικό Αλγόριθμο με Επιλογή Ρουλέτας και

\item στη συνιστώσα Επίδοσης, η οποία διατηρείται αμετάβλητη.
\end{itemize}

Παρουσιάσαμε, σε ένα πρώτο επίπεδο, τον κύκλο εκπαίδευσης του GMl-ASLCS$_{\:0}$ και τις τροποποιήσεις που ήταν αναγκαίες, ώστε αυτός να είναι ικανός για πολυκατηγορική ταξινόμηση. Η παραπάνω περιγραφή και ανάλυση ήταν αναγκαία, ώστε ο αναγνώστης να λάβει τις απαραίτητες γνώσεις για τον τρόπο λειτουργίας ενός πολυκατηγορικού ΜαΣΤ, αλλά και για την κατανόηση των λόγων για τις τροποποιήσεις που επιφέραμε στον αρχικό GMl-ASLCS (GMl-ASLCS$_{\:0}$).

Στη συνέχεια, παρουσιάσαμε τον κύκλο εκπαίδευσης του GMl-ASLCS, του τροποποιημένου αλγορίθμου ΜαΣΤ που αποτελεί το κεντρικό θέμα αυτής της εργασίας και εμβαθύναμε στα επιμέρους τμήματα και συνιστώσες του και στις εσωτερικές διεργασίες τους. 

Συγκεκριμένα, στη συνιστώσα Εξερεύνησης ανακαλύψαμε πως ο τελεστής Διασταύρωσης Ενός Σημείου επιβραδύνει σε ένα βαθμό την εκπαιδευτική διαδικασία στα πραγματικά σύνολα πολυκατηγορικών δεδομένων, λόγω της, σχεδόν νομοτελειακής, μεταφοράς του συνόλου των αποφάσεων για ετικέτες από τους κανόνες-γονείς προς τους απογόνους τους. Για αυτό το λόγο προτείναμε την εφαρμογή μίας πρωτότυπης μεθόδου διασταύρωσης, του τελεστή Διασταύρωσης Δύο Τμημάτων, ο οποίος προσιδιάζει στη φύση των πολυκατηγορικών προβλημάτων. Για κάθε σχηματιζόμενο ανά δείγμα $i$ και ετικέτα $l$ Correct Set, η Διασταύρωση Δύο Τμημάτων δρα μεταφέροντας από τους γονείς στους απογόνους μόνο την συγκεκριμένη απόφαση για την $l$, η οποία λόγω της φύσης του συνόλου Correct Set, συμφωνεί με την απόφαση του $i$ για την $l$. Με αυτό τον τρόπο, επαναληπτικά, μειώνουμε τη συμμετοχή λανθασμένων αποφάσεων για ετικέτες στο τμήμα απόφασης των κανόνων και επιτυγχάνουμε τη γρηγορότερη σύγκλιση του Γενετικού Αλγορίθμου και τη συνολική αύξηση της προβλεπτικής ικανότητας του μοντέλου που κατασκευάζει ο GMl-ASLCS.

Ωστόσο, η προβλεπτική ικανότητα των κανόνων του ΜαΣΤ δεν είναι από μόνη της μία ικανή συνθήκη για αποτελεσματική ταξινόμηση από το προβλεπτικό μοντέλο που κατασκευάζει ο GMl-ASLCS. Ταυτόχρονα, το μοντέλο θα πρέπει να μπορεί να ταξινομεί με ακρίβεια δείγματα με τα οποία δεν έχει εκπαιδευτεί. Με άλλα λόγια, οι κανόνες ενός ΜαΣΤ θα πρέπει να είναι ικανοί να γενικεύουν με ακρίβεια βάσει των δειγμάτων με τα οποία εκπαιδεύεται το ΜαΣΤ, ώστε να μπορούν να καλύπτουν αταξινόμητα δείγματα από το σύνολο ελέγχου. Για αυτό το σκοπό, προτείναμε την εφαρμογή μίας μεθόδου που αφαιρεί από τον πληθυσμό τους κανόνες εκείνους που καλύπτουν το μικρότερο αριθμό δειγμάτων μέσα σε κάθε Match Set και που έχουν τη χαμηλότερη καταλληλότητα ανάμεσα στους κανόνες του Match Set που καλύπτουν τον ίδιο αριθμό δειγμάτων. Για να καταστεί δικαιότερο το παραπάνω σχήμα διαγραφής, απαιτήσαμε από τους κανόνες να έχουν εξετάσει τουλάχιστον μία φορά κάθε δείγμα του συνόλου εκπαίδευσης ως προς την ικανότητα κάλυψής του.

Ακόμα, παρατηρήσαμε πως στον υπολογισμό της καταλληλότητας ενός κανόνα, ο GMl-ASLCS$_{\:0}$ συμπεριφέρεται με ισότιμο τρόπο στις σαφείς συμφωνίες και τις αδιαφορίες ως προς την απόφαση για μία ετικέτα, κάτι που ενέχει κινδύνους τόσο για την εξελικτική διαδικασία, όσο και για την τελική ταξινόμηση των δειγμάτων του συνόλου ελέγχου. Για να αποθαρρύνουμε το ΜαΣΤ από το να συσσωρεύουν οι κανόνες του αδιαφορίες στο τμήμα απόφασής τους, εισήγαμε ένα σχήμα έκπτωσης (ή τιμωρίας) για κάθε ετικέτα για την οποία ένας κανόνας δεν αποφασίζει με σαφήνεια.

Η τελευταία ριζοσπαστική μας κίνηση είχε ως αφετηρία την παρατήρηση του φαινομένου της υπερ-παραγωγής κανόνων μηδενικής κάλυψης, δηλαδή κανόνων που είναι ανίκανοι να ταξινομήσουν έστω και ένα του συνόλου εκπαίδευσης, στα σύνολα $enron$ και $medical$. Παρατηρήσαμε τη συμπιεστική δράση του παραπάνω φαινομένου, όσον αφορά στη διαφορά του αριθμού των κανόνων που επιθυμούμε να συγκρατεί το ΜαΣΤ ως πληθυσμό και την πραγματική του τιμή, αλλά και την επίδρασή του στις επιδόσεις του ΜαΣΤ. Η κίνησή μας ήταν να εξαλείψουμε την παρουσία κανόνων μηδενικής κάλυψης στον πληθυσμό του ΜαΣΤ, με τον άμεσο έλεγχο κάλυψης δειγμάτων για κάθε απόγονο που δημιουργείται από το Γενετικό Αλγόριθμο, ώστε να παρέχουμε ένα έρεισμα για την ομαλότητα της εξελικτικής διαδικασίας και να μπορούμε να προσδιορίσουμε με ακρίβεια τις παραμέτρους της εκπαίδευσης, ώστε στα πειράματα που διενεργούμε ο GMl-ASLCS να έχει ελέγξιμη συμπεριφορά και βέλτιστες επιδόσεις.

Γενικότερα, η εργασία στον πολυκατηγορικό χώρο με τη χρήση ΜαΣΤ, εκτός από την εισαγωγή των παραπάνω λειτουργιών, κυοφόρησε μία σειρά από παρατηρήσεις και συμπεράσματα, όπως το γιατί δεν πρέπει κανόνες που αδιαφορούν για μία ετικέτα να συμμετέχουν στο αντίστοιχο Correct Set το οποίο σχηματίζεται για αυτήν, το πώς επηρεάζει τη συμπεριφορά ενός ΜαΣΤ η μεταβολή του κατωφλίου εμπειρίας $\theta_{exp}$ και των βασικών παραμέτρων του $(\abs{I}, \theta_{GA}, maxPopulationSize, P_{\#A})$, ή ακόμα το πώς ο υπολογισμός της παραμέτρου $cs$ προκαλεί την υπερ-εκτίμηση και υπο-εκτίμησή της πραγματικής της τιμής για κάθε κανόνα, ανάλογα με την εμπειρία του και το χρονικό σημείο δημιουργίας του.

Στη συνέχεια, προτείναμε μερικές τροποποιήσεις σε επιμέρους τμήματα του GMl-ASLCS που θεωρήσαμε ως σκόπιμες και άξιες προς διερεύνηση της ικανότητάς τους να τον βοηθήσουν στο έργο της ταξινόμησης. Μεταξύ αυτών βρίσκεται η μεταβολή του τρόπου υπολογισμού της καταλληλότητας ενός κανόνα, η τροποποίηση της μεθόδου υπολογισμού των πιθανοτήτων διαγραφής των κανόνων μέσω επιλογής ρουλέτας και η αρχικοποίηση του πληθυσμού μέσω ομαδοποίησης.

Στο πειραματικό μέρος της εργασίας, πραγματοποιήσαμε μία πρώτη αξιολόγηση της συμπεριφοράς του GMl-ASLCS αλλά και της συνεισφοράς των τεσσάρων αρθρωτών τροποποιήσεων που επιφέραμε στον αρχικό GMl-ASLCS, σε τέσσερα τεχνητά σύνολα δεδομένων. Οι τροποποιήσεις αυτές αφορούν στη μεταβολή του τρόπου υπολογισμού των πιθανοτήτων που ανατίθενται στους κανόνες του πληθυσμού, στον τελεστή Διασταύρωσης Δύο Τμημάτων, στην αφαίρεση κανόνων του πληθυσμού μέσω της επιλεκτικής διαγραφής κανόνων από τα Match Sets και στην έκπτωση καταλληλότητας των κανόνων για κάθε παρουσία αδιαφορίας στο τμήμα απόφασής τους. Σε γενικές γραμμές, η συνεισφορά κάθε επιμέρους λειτουργίας που εισάγαμε ή τροποποιήσαμε αποτιμάται θετικά πάνω στα τέσσερα τεχνητά σύνολα δεδομένων ($mlPosition_{7}$, $mlIdentity_{7}$, $adder_{7}^{3}$ και $adder_{7}^{24}$), με την κάθε μία να βελτιώνει τις επιδόσεις του GMl-ASLCS$_{\:0}$. Εξαίρεση αποτελεί η υλοποίηση της τιμωρίας κανόνων για την παρουσία αδιαφοριών στο τμήμα απόφασης των κανόνων, όταν η πολυκατηγορική πυκνότητα ενός συνόλου δεδομένων εμφανίζει υψηλές τιμές. 

Παρ' όλα αυτά, η άθροιση των τεσσάρων παραπάνω συνιστωσών στον ορισμό του GMl-ASLCS τον κάνουν να εμφανίζει συνολικά βελτιωμένες επιδόσεις σε σχέση με αυτές του GMl-ASLCS$_{\:0}$ και στα τέσσερα παραπάνω τεχνητά σύνολα, για τη μετρικής της ακρίβειας. Επιπρόσθετα, η ικανότητα ανάπτυξης του Χάρτη Βέλτιστων Αποφάσεων (ΧΒΑ) από τον GMl-ASLCS βελτιώνεται, μόνο για τα σύνολα εκείνα στα οποία δεν υπάρχει παρουσία αδιαφοριών για ετικέτες στους κανόνες του. Λόγω της τιμωρίας της καταλληλότητας των κανόνων που διατηρούν ετικέτες με αδιαφορίες στο τμήμα απόφασής τους, ο GMl-ASLCS εμφανίζει μεγαλύτερη ικανότητα εύρεσης του ΧΒΑ για το σύνολο $mlPosition_{7}$, του οποίου ο ΧΒΑ περιλαμβάνει κανόνες που αποφασίζουν με σαφήνεια για τις ετικέτες του προβλήματος, αλλά αποτυγχάνει πλήρως να αναπτύξει τον ΧΒΑ του προβλήματος $mlIdentity_{7}$. Βεβαίως, αυτό είναι κάτι το αναμενόμενο και δεν επηρεάζει σε κανένα βαθμό τη συνολική προβλεπτική ικανότητα του GMl-ASLCS, καθώς η ικανότητα εύρεσης ενός ΧΒΑ είναι μόνο μία ικανή συνθήκη για αποτελεσματική ταξινόμηση, αλλά όχι αναγκαία.

Στη συνέχεια πραγματοποιήσαμε μία περισσότερο ρεαλιστική αξιολόγηση του GMl-ASLCS σε έξι, διαδεδομένα στη βιβλιογραφία, πραγματικά σύνολα πολυκατηγορικών δεδομένων, τα $music$, $yeast$, $genbase$, $scene$, $medical$ και $enron$. Συγκρίναμε τις επιδόσεις του GMl-ASLCS με αυτές του GMl-ASLCS$_{\:0}$ και τριών state-of-the-art πολυκατηγορικών αλγορίθμων (οι οποίοι δεν ανήκουν στην οικογένεια των ΜαΣΤ) στα παραπάνω σύνολα δεδομένων και βρήκαμε πως ο GMl-ASLCS κατατάσσεται πρώτος ανάμεσά τους για τη μετρική της ακρίβειας και δεύτερος για τη μετρικής της ακριβούς ορθότητας. Επιπρόσθετα, ο GMl-ASLCS εμφανίζει στατιστικά σημαντικές διαφορές στις επιδόσεις του με βάση την ακρίβεια σε σχέση με τον GMl-ASLCS$_{\:0}$. Αντιθέτως, για τη μετρική της ακριβούς ορθότητας, βρέθηκε ότι για τους πέντε υπό μελέτη αλγόριθμους δεν υπάρχει στατιστικά σημαντική διαφορά στις επιδόσεις τους. 

Ακόμα, συγκρίναμε τις επιδόσεις του GMl-ASLCS, και του τροποποιημένου GMl-ASLCS που χρησιμοποιεί τον τελεστή Διασταύρωσης Ενός Σημείου και βρήκαμε πως αν και οι δυο αλγόριθμοι δεν εμφανίζουν στατιστικά σημαντική διαφορά όσον αφορά στις μετρικές της ακρίβειας και της ακριβούς ορθότητας, ο δεύτερος κατατάσσεται συστηματικά σε χαμηλότερες θέσεις από τον πρώτο, συνεπώς, στο πλαίσιο λειτουργίας του GMl-ASLCS, ο τελεστής Διασταύρωσης Δύο Τμημάτων κρίνεται καταλληλότερος και αποτελεσματικότερος από αυτόν που χρησιμοποιούσε ο GMl-ASLCS$_{\:0}$.

Όσον αφορά στο στόχο μας για αύξηση του αριθμού των δειγμάτων που καλύπτουν οι κανόνες του ΜαΣΤ, με ταυτόχρονη διατήρηση της προβλεπτικής του ικανότητας, η εισαγωγή του νέου τμήματος διαγραφής κανόνων με κριτήρια που τίθενται στα Match Sets φαίνεται ότι τον επιτυγχάνει, αυξάνοντας μάλιστα και τις επιδόσεις του GMl-ASLCS, όσον αφορά στις βαρυσήμαντες μετρικές της ακρίβειας και της ακριβούς ορθότητας, για τα έξι πραγματικά σύνολα πολυκατηγορικών δεδομένων που μελετήσαμε.

Επιπρόσθετα, αξιολογήσαμε την ικανότητα ταξινόμησης των τροποποιήσεων που προτείναμε στον ορισμό του GMl-ASLCS. Αποδείχθηκε ότι δύο από αυτές τις τροποποιήσεις βρίσκονται πολύ κοντά στις επιδόσεις του GMl-ASLCS και, συνεπώς, ότι αξίζει η περαιτέρω έρευνά και εντρύφηση στις εσωτερικές διεργασίες των τροποποιημένων GMl-ASLCS. Οι τροποποιήσεις αυτές αφορούν στην αρχικοποίηση του πληθυσμού του ΜαΣΤ μέσω της ομαδοποίησης του συνόλου εκπαίδευσης, με παραμέτρους $(\gamma, P_{\#A}, P_{\#L}) \equiv (0.2, 0, 0)$ και στο μη υπολογισμό των αδιαφοριών για ετικέτες στη συνάρτηση υπολογισμού της καταλληλότητας των κανόνων του ΜαΣΤ, με παραμέτρους $(\omega, \phi) \equiv (0,0)$. 

Τέλος, η εισαγωγή του τμήματος διαγραφής κανόνων από τα Match Sets και η πρόνοια για την απαγόρευση εισαγωγής κανόνων μηδενικής κάλυψης στον πληθυσμό των ΜαΣΤ, λόγω της ανεξαρτησίας τους από τον αριθμό των ετικετών του προβλήματος προς επίλυση, θεωρούμε ότι μπορεί να εφαρμοστεί με αντίστοιχη επιτυχία για την επίλυση προβλημάτων μονοκατηγορικής ταξινόμησης.



\section{Περιορισμοί και Πλεονεκτήματα των ΜαΣΤ}
Οι αλγόριθμοι πολυκατηγορικής ταξινόμησης με ΜαΣΤ φαίνεται ότι είναι ικανοί να παρέχουν ικανοποιητικότερα αποτελέσματα σε σχέση με τις ντετερμινιστικές προσεγγίσεις της βιβλιογραφίας. Υπάρχουν, όμως, και περιορισμοί στη λειτουργία τους και στα αποτελέσματα που εξάγουν. Μερικοί από αυτούς είναι:

\begin{itemize}
\item Ο χρόνος εκπαίδευσης των ΜαΣΤ είναι πολύ μεγαλύτερος σε σχέση με τις αιτιοκρατικές προσεγγίσεις και, μάλιστα, αυξάνει όσο αυξάνει το μέγεθος του συνόλου εκπαίδευσης, η πολυπλοκότητά του (ο αριθμός των γνωρισμάτων, ετικετών και η πολυκατηγορική του πυκνότητα του) και ο αριθμός των κανόνων που απαιτούνται για τη λύση του προβλήματος. Ενδεικτικά, σε υπολογιστή με $2.66$ GHz επεξεργαστικής ισχύος και $4$ GB μνήμη, οι χρόνοι εκπαίδευσης του GMl-ASLCS στα πραγματικά σύνολα δεδομένων που χρησιμοποιήθηκαν σε αυτή την εργασία ήταν $280$ λεπτά για το σύνολο $music$, $4120$ για το $yeast$, $668$ για το $genbase$, $350$ για το $scene$, $640$ για το $medical$ και $1100$ λεπτά για το $enron$, χρησιμοποιώντας αξιολόγηση $10$-πλης διασταυρωμένης επικύρωσης για τα $music$, $yeast$ και $enron$ και προϋπάρχοντα χωρισμό σε σύνολα εκπαίδευσης και ελέγχου για τα $scene$, $medical$ και $enron$. Επιπρόσθετα, ο χρόνος εύρεσης του κατωφλίου με εσωτερική αξιολόγηση στη συνιστώσα Επίδοσης αυξάνει με το μέγεθος των κανόνων και των ετικετών του προβλήματος, σε σημείο τέτοιο όπου σε σύνολα δεδομένων όπως το $enron$, ο χρόνος για την εύρεσή του να παίρνει $2$ ώρες.

\item Τα ΜαΣΤ διατηρούν ένα μεγάλο σύνολο παραμέτρων που πρέπει να ρυθμιστούν κατάλληλα ώστε να διεξάγουν αποτελεσματική ταξινόμηση. Σε συνδυασμό με την πολυπλοκότητα που εισάγει η ρύθμιση κάθε παραμέτρου ξεχωριστά και την αλληλεξάρτηση των παραμέτρων, όσον αφορά στη συμπεριφορά ενός ΜαΣΤ κατά την εκπαιδευτική διαδικασία, αλλά και την τελική του ικανότητα ταξινόμησης, ο χρόνος για την εξαγωγή ικανοποιητικών αποτελεσμάτων αυξάνει δραματικά.

\item Κάποια σύνολα δεδομένων, όπως το $scene$ και το $medical$ που εξετάσαμε σε αυτή την εργασία, απαιτούν πολύ περισσότερες επαναλήψεις μάθησης από ότι άλλα σύνολα. Η αραιότητα των δειγμάτων τους συνηγορεί στην αργή σύγκλιση του Γενετικού Αλγορίθμου και κάνει απαραίτητη την επέκταση της εκπαιδευτικής διαδικασίας και, άρα, του χρόνου εκπαίδευσης ενός ΜαΣΤ.

\item Το θεωρητικό υπόβαθρο που μπορεί να βοηθήσει στην πράξη την κατηγοριοποίηση με ΜαΣΤ είναι ακόμη μικρό, λόγω της στοχαστικής φύσης τους και το γεγονός πως συντελείται μικρότερο μέγεθος έρευνας προς την κατεύθυνση των ΜαΣΤ σε σχέση με άλλες μεθόδους κατηγοριοποίησης.

\item Η έλλειψη των απαραίτητων εργαλείων για την επαρκέστερη γνώση του τι πραγματικά συμβαίνει στο εσωτερικό των διεργασιών ενός ΜαΣΤ, λόγω βεβαίως και της στοχαστικής τους φύσης, κάνει δύσκολη τη διείσδυσή της ανάλυσής μας σε βάθος, ώστε να αποκτήσουμε μία πληρέστερη εικόνα για τα διάφορα τμήματά του και να προχωρήσουμε μακρύτερα την έρευνα στον τομέα τους.

\item Η αποτύπωση της πορείας των μετρικών αξιολόγησης ενός ΜαΣΤ και, άρα της προόδου του με το πέρασμα των επαναλήψεων μάθησης, μπορεί να γίνει μόνο σε σύνολα δεδομένων με μικρή πολυπλοκότητα, όπως στα τεχνητά σύνολα δεδομένων που εξετάσαμε. Η πορεία της μετρικής της ακρίβειας, για παράδειγμα, με ρύθμιση κατωφλίου με τη μέθοδο PCut, για κάθε επανάληψη μάθησης είναι χρονοβόρα σε τέτοιο βαθμό που τριπλασιάζει το χρόνο εκπαίδευσης. Ακόμα περισσότερο, η χρήση της εύρεσης κατωφλίου με τη μέθοδο εσωτερικής αξιολόγησης είναι απαγορευτική.

\end{itemize}



Αντιθέτως, όμως, με τις αιτιοκρατικές προσεγγίσεις της πολυκατηγορικής ταξινόμησης, τα ΜαΣΤ περιλαμβάνουν ένα μεγάλο φάσμα από διαφορετικές λειτουργίες, με χώρο για τη μελέτη, τη βελτίωσή και την παραλληλοποίησή τους, αλλά και πρόσφορο έδαφος για την εισαγωγή νέων τμημάτων των οποίων οι λειτουργίες βασίζονται στην ανάλυση των εσωτερικών διεργασιών και δομών ενός ΜαΣΤ. Η μη-αιτιοκρατική τους φύση μπορεί να φαίνεται ως εμπόδιο όσον αφορά στην αναλυτική προσιτότητα τους, ταυτόχρονα, όμως, είναι το στοιχείο εκείνο που τα κάνει ευέλικτα και εύρωστα. Η βασισμένη-στη-φύση προσέγγισή τους, με τη θεωρία της εξέλιξης των ειδών να αποτελεί τον πυλώνα της δημιουργίας λύσεων πάνω σε προβλήματα, καθιστά τα ΜαΣΤ πλήρως ουσιώδη και ικανά να προσεγγίσουν με ακρίβεια σχεδόν ό,τι πρόβλημα μπορεί να περιγραφεί με τη μορφή μίας συλλογής δεδομένων.
