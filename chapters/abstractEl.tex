\begin{center}
\centering

\vspace{0.5cm}
\centering
\textbf{\Large{Περίληψη}}

\vspace{1cm}

\end{center}
Με τον αυξανόμενο ρυθμό παραγωγής δεδομένων και πληροφοριών παγκοσμίως και τη στροφή της ανθρωπότητας προς την αυτοματοποίηση όλο και περισσότερων διαδικασιών της σύγχρονης ζωής, τις τελευταίες δεκαετίες έχει υπάρξει αυξημένο ενδιαφέρον για τη Μηχανική Μάθηση, έναν τομέα της Υπολογιστικής Νοημοσύνης. Ο τομέας αυτός ασχολείται με την ανάπτυξη μηχανών που μπορούν να μαθαίνουν από την εμπειρία, ώστε να αναλάβουν αυτές το γιγάντιο έργο της αυτοματοποίησης διεργασιών και δραστηριοτήτων - ένα έργο που πλέον καμία ανθρώπινη μονάδα δεν μπορεί να φέρει εις πέρας. Η αυτοματοποίηση αυτή αφορά στην πρόβλεψη, εξήγηση ή/και κατανόηση των υποκείμενων δεδομένων. Πολλά προβλήματα μπορούν να περιγραφούν από ένα σύνολο δεδομένων που συλλέχθηκαν κάποια συγκεκριμένη στιγμή και, έτσι, μία πληθώρα μεθόδων Μηχανικής Μάθησης μπορεί να βοηθήσει στην εξαγωγή εύκολα ερμηνεύσιμων μοντέλων, διαφόρων μέσων, όπως είναι οι κανόνες ή τα δένδρα αποφάσεων. Ωστόσο, ειδικά στην περίπτωση της πρόβλεψης, υπάρχουν ειδικές περιστάσεις, όπως όταν το πρόβλημα απαιτεί αλληλεπίδραση με κάποια άλλη οντότητα, όπου ο αριθμός των επιλογών μειώνεται και τα Μανθάνοντα Συστήματα Ταξινομητών γίνονται η καλύτερη, αν όχι η μόνη, λύση.

Τα Μανθάνοντα Συστήματα Ταξινομητών (ΜαΣΤ) ανήκουν σε μία κλάση συστημάτων Μηχανικής Μάθησης Βασισμένης στη Γενετική (ΜΜΒΓ), τα οποία είναι σχεδιασμένα ώστε να μπορούν να αντιμετωπίσουν τόσο σειριακά, όσο και ενός-βήματος προβλήματα απόφασης, χρησιμοποιώντας κανόνες ταξινόμησης. Η παρούσα εργασία εστιάζει σε προβλήματα ταξινόμησης και, πιο συγκεκριμένα, χρησιμοποιεί ΜαΣΤ τύπου Michigan για να αντιμετωπίσει προβλήματα πολυκατηγορικής φύσης.

Η πολυκατηγορική ταξινόμηση είναι μία διαδικασία Εξόρυξης Δεδομένων όπου κάθε δείγμα ενός συνόλου δεδομένων συσχετίζεται με περισσότερες από μία κατηγορίες που ονομάζονται ετικέτες. Πολυκατηγορικά δεδομένα εμφανίζονται σε αφθονία σε πραγματικά προβλήματα, όπως οι ιατρικές διαγνώσεις, οι κατηγοριοποιήσεις εγγράφων ή η συσχέτιση γονιδίων με βιολογικές λειτουργίες.

Η παρούσα εργασία βασίζεται και επεκτείνει την προσέγγιση της πολυκατηγορικής ταξινόμησης του αλγορίθμου GMl-ASLCS, ο οποίος με τη σειρά του επεκτείνει το πλαίσιο εποπτευόμενης μάθησης AS-LCS στον πολυκατηγορικό χώρο. Εδώ πρέπει να σημειώσουμε πως, από όσο γνωρίζουμε, η προσέγγιση της πολυκατηγορικής ταξινόμησης με χρήση ΜαΣΤ είναι από τις πρώτες στον αντίστοιχο χώρο.

Βασιζόμενοι στις παραπάνω μεθόδους, η προσέγγισή μας κινείται σε τρεις άξονες: i) την εμβάθυνση στις λειτουργίες ενός (πολυκατηγορικού) ΜαΣΤ, μελετώντας και αναλύοντας τις εσωτερικές του διαδικασίες, ii) την προσέγγιση του υπό μελέτη προβλήματος περισσότερο από τη σκοπιά του μηχανικού και λιγότερο από αυτή της Επιστήμης Υπολογιστών (με την έννοια ότι μελετούμε ευρύτερα τη συμπεριφορά διαφορετικών τμημάτων ενός ΜαΣΤ και των μεταβολών που επιφέρουν στη συμπεριφορά του οι επιμέρους μεταβολές των παραμέτρων που τη διέπουν), και iii) τη βελτίωση της συνολικής συμπεριφοράς του GMl-ASLCS βάσει των δύο παραπάνω αξόνων, τόσο ως προς τις μετρικές αξιολόγησης που χρησιμοποιούνται, όσο και ως προς τη συμπεριφορά των επιμέρους τμημάτων του.

Η προσήλωση μας στους τρεις παραπάνω άξονες έχει ως αποτέλεσμα την εξαγωγή μίας σειράς από παρατηρήσεις και την επινόηση και υιοθέτηση διορθωτικών και διαρθρωτικών δράσεων:
\begin{enumerate}
\item Εμβαθύνοντας στο Γενετικό Αλγόριθμο που χρησιμοποιούν τα ΜαΣΤ, προτείνουμε την υιοθέτηση ενός νέου τελεστή Διασταύρωσης, τον τελεστή Διασταύρωσης Δύο Τμημάτων, ο οποίος προσιδιάζει στη φύση των πολυκατηγορικών προβλημάτων
\item Για τη διεύρυνση του αριθμού των δειγμάτων ενός συνόλου δεδομένων που μπορούν να ταξινομήσουν οι κανόνες του ΜαΣΤ με ακρίβεια, προτείνουμε την εισαγωγή ενός νέου τμήματος διαγραφής κανόνων που εφαρμόζεται σε επιμέρους σύνολα κανόνων αντί για το σύνολο του πληθυσμού τους.
\item Αναλύοντας εσωτερικά τη συμπεριφορά του πρώτου GMl-ASLCS, ανακαλύπτουμε τις σοβαρές συνέπειες της διατήρησης κανόνων στον πληθυσμό του ΜαΣΤ οι οποίοι είναι ανίκανοι να ταξινομήσουν δείγματα που τους παρουσιάζονται και προτείνουμε την εφαρμογή μίας μεθοδολογίας που τις εξαλείφει.
\item Παρατηρούμε την αντίρροπη δράση της υπερ-συσσώρευσης μη σαφών αποφάσεων για ετικέτες στο τμήμα απόφασης των κανόνων και υιοθετούμε μία προσέγγιση που μειώνει σε λογικό βαθμό αυτές τις αποφάσεις.
\end{enumerate}

Τέλος, κάνουμε παρατηρήσεις πάνω στις διαφορετικές λειτουργίες ενός ΜαΣΤ (που μπορούν να χρησιμοποιηθούν για την εξαγωγή συμπερασμάτων, τόσο όσον αφορά σε πολυκατηγορικά όσο και σε μονοκατηγορικά ΜαΣΤ) και προτείνουμε μερικές τροποποιήσεις στον πλήρη ορισμό του GMl-ASLCS, με σκοπό την αύξηση των επιδόσεών του, όπως για παράδειγμα την αρχικοποίηση του πληθυσμού χρησιμοποιώντας την ομαδοποίηση των δειγμάτων του πολυκατηγορικού συνόλου με το οποίο εκπαιδεύεται ο GMl-ASLCS.

Μία αρχική αξιολόγηση του ανανεωμένου GMl-ASLCS πραγματοποιείται σε τρία πολυκατηγορικά τεχνητά προβλήματα, σχεδιασμένα ώστε να δοκιμάσουν τις επιδόσεις του σε διαφορετικά περιβάλλοντα, λιγότερο πολύπλοκα από ότι τα πραγματικά σύνολα δεδομένων. Όσον αφορά στα πραγματικά σύνολα πολυκατηγορικών δεδομένων, οι επιδόσεις του GMl-ASLCS συγκρίνονται με αυτές του πρωταρχικού GMl-ASLCS και των διαδεδομένων μεθόδων πολυκατηγορικής ταξινόμησης RA$k$EL-J48, Ml$k$NN και BR-J48. Σύμφωνα με τα αποτελέσματα της πειραματικής διαδικασίας, η παρούσα έκδοση του GMl-ASLCS κατατάσσεται πρώτη ανάμεσά τους, επιδεικνύοντας, επιπλέον, στατιστικά σημαντικές διαφορές σε σχέση με τον προκάτοχό του.

Τέλος, εξετάζουμε μεμονωμένα την επίδραση που είχαν οι τέσσερις λειτουργίες που μεταβάλαμε στην αρχική έκδοση του GMl-ASLCS στα τεχνητά πολυκατηγορικά προβλήματα που προαναφέραμε και καταγράφουμε τις επιδόσεις των τροποποιημένων εκδόσεων GMl-ASLCS στα πραγματικά σύνολα δεδομένων που χρησιμοποιήσαμε. Από αυτές, ξεχωρίζουμε μία έκδοση του GMl-ASLCS που χρησιμοποιεί ομαδοποίηση για την αρχικοποίηση του πληθυσμού των κανόνων του και μία που δεν λαμβάνει υπόψη τις μη σαφείς αποφάσεις κανόνων στον υπολογισμό της καταλληλότητάς τους.

Συνολικά, η παρούσα εργασία έχει ως αποτέλεσμα τη βελτίωση της συμπεριφοράς επιμέρους τμημάτων του αλγορίθμου GMl-ASLCS, αλλά και της συνολικής συμπεριφοράς και επίδοσής του, όσον αφορά στις μετρικές που χρησιμοποιούνται για την αξιολόγηση του μοντέλου που αναπτύσσει και την αύξηση του αριθμού των δειγμάτων που μπορεί να ταξινομήσει με ακρίβεια.

Το κείμενο της παρούσας εργασίας βρίσκεται αναρτημένο στο διαδίκτυο στον παρακάτω
σύνδεσμο: \url{https://github.com/li9i/auth_thesis}. Η κωδικοποίηση του
αναφερθέντων πλαισίων και οι προσωμοιώσεις της συμπεριφοράς τους έγιναν σε JAVA
και είναι αναρτημένες στον παρακάτω σύνδεσμο: \url{https://github.com/li9i/mlaslcs}.
