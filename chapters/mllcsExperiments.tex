\chapter{Πειράματα Πραγματικών Συνόλων Δεδομένων}
\label{realWorldDatasetsExps}
Στη διεθνή βιβλιογραφία υπάρχει πληθώρα διαφορετικών συνόλων πολυκατηγορικών δεδομένων που χρησιμοποιούνται ως μέσα δοκιμής των αλγορίθμων πολυκατηγορικής ταξινόμησης. Επιπλέον, στο \cite{allamanis11} παρουσιάζονται οι πρώτες επιδόσεις του GMl-ASLCS$_{\:0}$, σε σχέση με έξι πολυκατηγορικά σύνολα δεδομένων. 

Το παρόν κεφάλαιο χωρίζεται σε δύο μέρη: στο πρώτο μέρος εστιάζουμε στα ίδια έξι σύνολα δεδομένων, παρουσιάζουμε τις διαφορές στις επιδόσεις ανάμεσα στον GMl-ASLCS$_{\:0}$ και τον GMl-ASLCS για τα σύνολα αυτά και συγκρίνουμε τις επιδόσεις του GMl-ASLCS με τρεις γνωστούς αλγορίθμους από τη βιβλιογραφία της πολυκατηγορικής ταξινόμησης. Στο δεύτερο μέρος διενεργούμε πειράματα α) για να εξακριβωθούν οι επιδόσεις των διαφόρων τροποποιήσεων που μπορούμε να επιφέρουμε στον GMl-ASLCS, που παρουσιάστηκαν στην Παρ. \ref{sec:alterations} και β) για να προσδιοριστεί η επίδραση των δύο πρωτοτυπιών που εισάγει αυτή η εργασία, του τελεστή Διασταύρωσης Δύο Τμημάτων και της λειτουργίας διαγραφής κανόνων με κριτήρια κάλυψης και καταλληλότητας από τα Match Sets, στην απόδοση του GMl-ASLCS.


\section{Υπό Μελέτη Σύνολα Δεδομένων}

\begin{table}
\begin{center}
\caption{Συνοπτικά χαρακτηριστικά των πραγματικών συνόλων δεδομένων.}
\label{table:realDatasetsChar}
  \begin{tabular}{l|lllllll}
    \hline\\ [-2ex]
   Dataset	         & $\abs{D}$  & $\abs{L}$ & $M$   & $LC$ & $LD$ & $P_{DIST}$ & $P_{MaxF}$ \\
    \hline \\ [-2ex]
    Music   & $593$  & $6$        	& $72n$   & $1.87$ & $0.31$ & $0.046$      & $0.137$      \\
    Yeast   & $2417$ & $14$       	& $103n$  & $4.24$ & $0.30$ & $0.082$      & $0.098$      \\
    Genbase & $661$  & $27$        	& $1185b$ & $1.25$ & $0.05$ & $0.048$      & $0.257$      \\
    \hline
    Scene   & $2407$ & $6$         	& $294n$  & $1.07$ & $0.18$ & $0.006$      & $0.168$      \\
    Medical & $978$  & $45$        	& $1449b$ & $1.25$ & $0.03$ & $0.096$      & $0.158$      \\
    Enron   & $1702$ & $53$        	& $1001b$ & $3.38$ & $0.06$ & $0.442$      & $0.096$      \\
    \hline
  \end{tabular}
\end{center}
\end{table}

Τα χαρακτηριστικά των πραγματικών συνόλων δεδομένων που χρησιμοποιήθηκαν για την πειραματική αξιολόγηση των GMl-ASLCS$_{\:0}$, GMl-ASLCS και των τροποποιήσεών του τελευταίου παρουσιάζονται στον Πίνακα \ref{table:realDatasetsChar}, όπου $\abs{D}$ είναι ο αριθμός δειγμάτων του συνόλου δεδομένων, $\abs{L}$ ο αριθμός των ετικετών των δειγμάτων του και $M$ ο αριθμός των γνωρισμάτων του, με το \emph{b} να σηματοδοτεί τα δυαδικά γνωρίσματα και το \emph{n} αριθμητικά. Οι στήλες $LC$, $LD$, $P_{DIST}$ και $P_{MaxF}$ αναφέρονται αντίστοιχα στην πολυκατηγορική πληθικότητα, στην πολυκατηγορική πυκνότητα, στο ποσοστό μοναδικότητας των συνδυασμών ετικετών των δειγμάτων και στο ποσοστό συχνότερου συνδυασμού ετικετών (βλ. Παρ. \ref{subsec:mlDatasetsProperties}).

Τα υπό μελέτη έξι σύνολα δεδομένων είναι διαθέσιμα στο δικτυακό χώρο της εφαρμογής πολυκατηγορικής ταξινόμησης Mulan\footnote{http://mulan.sourceforge.net/datasets.html} και είναι τα εξής:

\begin{description}
\item \textbf{Music} Είναι ένα μικρό σύνολο δεδομένων \cite{trohidis2008multilabel} όπου κατηγοριοποιούνται κομμάτια μουσικής σε έξι πιθανά συναισθήματα (sad-lonely, angry-aggressive, amazed-surprised, relaxing-calm, quiet-still, happy-pleased).
\item \textbf{Yeast} Είναι ένα ευρέως χρησιμοποιούμενο σύνολο δεδομένων \cite{elisseeff2002kernel}, όπου γονίδια συσχετίζονται με $14$ βιολογικές λειτουργίες-ετικέτες.
\item \textbf{Genbase} Είναι ακόμη ένα βιολογικό σύνολο δεδομένων \cite{diplaris2005protein}, παρόμοιο με το yeast, το οποίο καταγράφει τη συσχέτιση γονιδίων με 27 λειτουργίες-ετικέτες.
\item \textbf{Scene} Αναφέρεται στην πολυκατηγορική ταξινόμηση τοπίων (scenes) σε έξι πιθανές κατηγορίες (beach, sunset, field, fall-foliage, mountain, urban) \cite{boutell2004learning}.
\item \textbf{Medical} Αποτελεί ένα σύνολο δεδομένων ιατρικών κειμένων που δημιουργήθηκε για το Computational Medicine Centers 2007 Medical Natural Language Processing Challenge \cite{pestian2007shared}. Κάθε έγγραφο αποτελείται από μία περιγραφή των συμπτωμάτων ενός ασθενή, ενώ οι ετικέτες αντιστοιχούν σε κωδικούς ασφάλισης.
\item \textbf{Enron} Αποτελεί το σύνολο δεδομένων που συλλέχθηκαν από τα ηλεκτρονικά μηνύματα της εταιρίας Enron στα πλαίσια του Enron Email Analysis\footnote{http://bailando.sims.berkeley.edu/enron\_email.html}, μετά το σκάνδαλο Enron το 2001. Αν και στο αρχικό σύνολο δεδομένων έχει γίνει ιεραρχική κατηγοριοποίηση των δεδομένων, στο τελικό σύνολο χρησιμοποιούνται μόνο οι ετικέτες-φύλλα.
\end{description}





\section{Πειράματα σε Πραγματικά Σύνολα Δεδομένων}
Τα αναλυτικά αποτελέσματα από την αξιολόγηση του GMl-ASLCS πάνω στα έξι παραπάνω πραγματικά σύνολα δεδομένων παρατίθεται στους Πίνακες \ref{table:musicEvals} έως \ref{table:enronEvals} για κάθε σύνολο δεδομένων. Κάθε πίνακας αναφέρεται σε ένα σύνολο δεδομένων και περιέχει τις τιμές των μετρικών της Ακρίβειας, Ανάκλησης, Απώλειας Hamming και Ακριβούς Ορθότητας. Για κάθε μετρική αξιολόγησης εμφανίζονται τρία αποτελέσματα, ένα για μία από τις στρατηγικές συμπερασμού, IVal, PCut και Best, εφόσον αυτές είναι εφαρμόσιμες. Για την πληρέστερη αξιολόγηση του GMl-ASLCS, εκτός από τον GMl-ASLCS$_{\:0}$, παρατίθενται και οι επιδόσεις των state-of-art αλγορίθμων πολυκατηγορικής ταξινόμησης:

\begin{description}
\item \textbf{BR-J48} Αποτελεί το μετασχηματισμό \emph{BR} των προβλημάτων με χρήση $\abs{L}$ δυαδικών ταξινομητών J48.
\item \textbf{RA$k$EL-J48} Αποτελεί έναν αλγόριθμο μετασχηματισμού σε \emph{k} διαφορετικά σύνολα ετικετών \cite{tsoumakas2007random}. Και εδώ ως ταξινομητής βάσης χρησιμοποιείται ο J48.
\item \textbf{Ml$k$NN} Αποτελεί μία τροποποίηση του κλασικού ταξινομητή $k$NN ($k$ Nearest Neighbors) που μπορεί να προβλέπει απευθείας δείγματα που ανήκουν σε περισσότερες από μία κατηγορίες.
\end{description}

Για τα πειράματα στα σύνολα δεδομένων music, yeast και genbase χρησιμοποιήθηκε αξιολόγηση 10-πλης διασταυρωμένης επικύρωσης (10-fold cross validation), ενώ για τα scene, medical και enron χρησιμοποιήθηκε ο προϋπάρχων (στο διαδικτυακό χώρο της εφαρμογής Mulan) χωρισμός τους σε σύνολα εκπαίδευσης και ελέγχου. Για τα τρία τελευταία σύνολα δεδομένων, τα σύνολα εκπαίδευσης και ελέγχου πάρθηκαν από το διαδικτυακό χώρο της εφαρμογής Mulan.

\subsection{Πειραματική Μεθοδολογία}
\label{subsec:realDatasetsExperiments}
Όσον αφορά στην επιλογή των τιμών των παραμέτρων των επιμέρους λειτουργιών του GMl-ASLCS, εστιάσαμε την προσοχή μας σε ένα υποσύνολο των διαθέσιμων παραμέτρων. Οι παράμετροι αυτές ρυθμίστηκαν ανά πρόβλημα, σε πρώτη φάση με βάση τους εμπειρικούς κανόνες που αναφέρονται σε παρατηρήσιμα χαρακτηριστικά των υπό μελέτη συνόλων δεδομένων και, σε δεύτερη, μέσω μιας διαδικασίας δοκιμής και σφάλματος. Οι εμπειρικοί κανόνες που χρησιμοποιήθηκαν προκύπτουν από τις εξής παρατηρήσεις:

\begin{description}
\item \textbf{Αριθμός και είδος Γνωρισμάτων} Ο αριθμός και το είδος των γνωρισμάτων των δειγμάτων ενός συνόλου δεδομένων επηρεάζει τον απαιτούμενο αριθμό κανόνων, ώστε να εξασφαλιστεί η αποτελεσματικότητα της εξελικτικής διαδικασίας. Επιπλέον, η ύπαρξη αριθμητικών γνωρισμάτων επιδεινώνει την κατάσταση, αφού δημιουργεί μεγαλύτερους χώρους αναζήτησης.
\item \textbf{Πολυπλοκότητα Προβλήματος} Αν και δε διαθέτουμε κάποιο σαφές μέσο υπολογισμού αυτού του μεγέθους, προβλήματα με αυξημένη πολυπλοκότητα, δηλαδή με μεγάλο αριθμό δειγμάτων, γνωρισμάτων, ετικετών, πολυκατηγορικής πυκνότητας, ή/και διακριτών συνδυασμών ετικετών, απαιτούν συνήθως μεγαλύτερα μεγέθη πληθυσμών κανόνων και μεγαλύτερες τιμές πιθανότητας γενίκευσης (γνωρισμάτων ή/και ετικετών).
\item \textbf{Ανισορροπία ετικετών} Η ανισορροπία στη συχνότητα εμφάνισης των διακριτών συνδυασμών ετικετών επηρεάζει τη συμπεριφορά των ΜαΣΤ, αφού δημιουργούνται κανόνες με μεγάλες διαφορές ως προς την εμπειρία και την κάλυψη δειγμάτων και, επομένως, στην ποιότητα προσέγγισης των παραμέτρων τους. Για την προστασία εκείνων των κανόνων που καλύπτουν τα υπο-αντιπροσωπευόμενα δείγματα του χώρου αναζήτησης των ετικετών είναι αναγκαία α) η ακριβής ρύθμιση της πιθανότητας διαγραφής των κανόνων με βάση την εμπειρία και του κατωφλίου της εμπειρίας $\theta_{del}$, β) η ρύθμιση του κατωφλίου εμπειρίας $\theta_{exp}$ όσον αφορά στην έκπτωση με βάση την εμπειρία στη συνιστώσα εξερεύνησης και γ) η ρύθμιση του ρυθμού ενεργοποίησης του Γενετικού Αλγορίθμου $\theta_{GA}$, ώστε να εφαρμόζεται σε σύνολα κανόνων με ικανοποιητικές προσεγγίσεις των παραμέτρων ποιότητάς τους.
\item \textbf{Βαθμός πλήρωσης του συνόλου δεδομένων} Σύνολα δεδομένων με μεγάλο αριθμό γνωρισμάτων αλλά μικρό αριθμό δειγμάτων, σε σχέση με το μέγιστο αριθμό συνδυασμών των τιμών των γνωρισμάτων, απαιτούν προσεκτική ρύθμιση του μεγέθους του πληθυσμού και του ρυθμού ενεργοποίησης του Γενετικού Αλγορίθμου, ενώ χρειάζονται μεγάλες τιμές της πιθανότητας γενίκευσης γνωρισμάτων και μικρές της πιθανότητας γενίκευσης ετικετών (χαρακτηριστικά παραδείγματα αποτελούν τα σύνολα δεδομένων enron, medical και scene).
\end{description}

Οι συγκεκριμένες παράμετροι που χρησιμοποιήθηκαν ανά πείραμα παρατίθενται στον Πίνακα \ref{table:realDatasetsParams}. Κάθε δείγμα παρουσιάζεται στο σύστημα I φορές, με συνολικό αριθμό επαναλήψεων μάθησης I $\cdot \abs{D}$. $P_{\#A}$ και $P_{\#L}$ είναι οι πιθανότητες γενίκευσης γνωρισμάτων και ετικετών, αντίστοιχα, που χρησιμοποιούνται από τη λειτουργία κάλυψης, ενώ $\abs{P}$ είναι ο συνολικός αριθμός (άνω όριο) μικρο-κανόνων που επιθυμούμε να συγκρατήσει το ΜαΣΤ ως πληθυσμό.

\begin{table}[!h]
\begin{center}
    \caption{Παράμετροι πειραμάτων του GMl-ASLCS.}
	\label{table:realDatasetsParams}
    \begin{tabular}{l|rrrll}

	\hline \\ [-2ex]
    Σύνολο δεδομένων	      & I      & $\abs{P}$ & $\theta_{GA}$ & $P_{\#A}$ & $P_{\#L}$ \\
    \hline \\ [-2ex]
    music   & $500$  & $5000$    & $2000$        & $0.80$    & $0.01$    \\
    yeast   & $500$  & $18000$   & $4000$        & $0.85$    & $0.01$    \\
    genbase & $500$  & $12000$   & $2000$        & $0.40$    & $0.10$     \\
    scene   & $2500$ & $9000$    & $300$         & $0.99$    & $0.10$     \\
    medical & $4000$ & $2500$    & $2000$        & $0.99$    & $0.10$     \\
    enron   & $600$  & $25000$   & $2000$        & $0.99$    & $0.10$     \\
    \hline 
    \end{tabular}
    \end{center}
\end{table}

Τέλος, οι κοινές παράμετροι που χρησιμοποιήθηκαν είχαν τις εξής τιμές: $\mu=0.04$, $\chi = 0.8$, $\nu=10$, $\beta = 0.2$, $\omega=0.9$ και $\phi=1$. 








\begin{landscape}
\begin{table}
\begin{center}
\caption[Αποτελέσματα στο σύνολο δεδομένων music.]{Αποτελέσματα του GMl-ASLCS, του προκατόχου του, GMl-ASLCS$_{\:0}$, και των αλγορίθμων πολυκατηγορικής ταξινόμησης της βιβλιογραφίας οι οποίοι δεν ανήκουν στην οικογένεια των ΜαΣΤ για τις μετρικές της Ακρίβειας, της Ανάκλησης, της Απώλειας Hamming και της Ακριβούς Ορθότητας στο σύνολο δεδομένων music.}
\label{table:musicEvals}
\begin{tabular}{ccccccccccccccccc}
 & \multicolumn{3}{c}{Ακρίβεια (\%)} & & \multicolumn{3}{c}{Ανάκληση (\%)} & & \multicolumn{3}{c}{Απώλεια Hamming (\%)} & & \multicolumn{3}{c}{Ακριβής Ορθότητα (\%)}
\\ 
\cline{2-4} \cline{6-8} \cline{10-12} \cline{14-16}
& \textsc{PCut} & \textsc{IVal} & \textsc{Best} & & \textsc{PCut} & \textsc{IVal} & \textsc{Best} & & \textsc{PCut} & \textsc{IVal} & \textsc{Best} & & \textsc{PCut} & \textsc{IVal} & \textsc{Best}\\ \hline
GMl-ASLCS$_{\:0}$ & $49.57$ & $50.05$ & $43.44$ & & $61.14$ & $63.51$ & $52.19$ & & $23.6$ & $24.05$ & $26.48$ & & $23.78$ & $23.11$ & $22.42$ \\ 
GMl-ASLCS 		  & $\textbf{58.68}$ & $\textbf{60.47}$ & $\textbf{48.20}$ & & $\textbf{70.24}$ & $\textbf{74.42}$ & $\textbf{60.80}$ & & $\textbf{19.07}$	& $\textbf{18.77}$ & $\textbf{25.11}$	& & $\textbf{33.75}$ & $\textbf{34.39}$	& $\textbf{24.31}$ \\
\hline
\hline
BR-J48 			  & - & $46.23$ & - & & - & $59.94$ & - & & -	& $24.74$ & -	& & - & $18.38$	& - \\
RA$k$EL-J48 	  & - & $50.91$ & - & & - & $62.73$ & - & & -	& $21.81$ & -	& & - & $24.78$	& - \\
Ml$k$NN 		  & - & $53.26$ & - & & - & $60.50$ & - & & -	& $19.51$ & -	& & - & $28.31$	& - \\
\hline

\end{tabular} 
\end{center}
\end{table}


\begin{table}
\begin{center}
\caption[Αποτελέσματα στο σύνολο δεδομένων yeast.]{Αποτελέσματα του GMl-ASLCS, του προκατόχου του, GMl-ASLCS$_{\:0}$, και των αλγορίθμων πολυκατηγορικής ταξινόμησης της βιβλιογραφίας οι οποίοι δεν ανήκουν στην οικογένεια των ΜαΣΤ για τις μετρικές της Ακρίβειας, της Ανάκλησης, της Απώλειας Hamming και της Ακριβούς Ορθότητας στο σύνολο δεδομένων yeast.}
\label{table:yeastEvals}
\begin{tabular}{ccccccccccccccccc}
 & \multicolumn{3}{c}{Ακρίβεια (\%)} & & \multicolumn{3}{c}{Ανάκληση (\%)} & & \multicolumn{3}{c}{Απώλεια Hamming (\%)} & & \multicolumn{3}{c}{Ακριβής Ορθότητα (\%)}
\\ 
\cline{2-4} \cline{6-8} \cline{10-12} \cline{14-16}
& \textsc{PCut} & \textsc{IVal} & \textsc{Best} & & \textsc{PCut} & \textsc{IVal} & \textsc{Best} & & \textsc{PCut} & \textsc{IVal} & \textsc{Best} & & \textsc{PCut} & \textsc{IVal} & \textsc{Best}\\ \hline
GMl-ASLCS$_{\:0}$ & $46.09$ & $45.51$ & $39.01$ & & $60.83$ & $65.69$ & $44.28$ & & $23.91$ & $25.16$ & $\textbf{23.90}$ & & $6.33$ & $5.01$ & $5.79$ \\ 
GMl-ASLCS 		  & $\textbf{51.92}$ & $\textbf{51.67}$ & $\textbf{41.49}$ & & $\textbf{64.20}$ & $\textbf{69.06}$ & $\textbf{55.15}$ & & $\textbf{21.23}$	& $22.30$ & $26.88$	& & $\textbf{13.07}$ & $11.36$	& $\textbf{8.80}$ \\
\hline
\hline
BR-J48 			  & - & $43.95$ & - & & - & $57.84$ & - & & -	& $24.54$ & -	& & - & $6.83$	& - \\
RA$k$EL-J48 	  & - & $48.74$ & - & & - & $62.89$ & - & & -	& $22.58$ & -	& & - & $11.71$	& - \\
Ml$k$NN 		  & - & $51.62$ & - & & - & $59.13$ & - & & -	& $\textbf{19.33}$ & -	& & - & $\textbf{18.74}$	& - \\
\hline

\end{tabular} 
\end{center}
\end{table}
\end{landscape}




\begin{landscape}
\begin{table}
\begin{center}
\caption[Αποτελέσματα στο σύνολο δεδομένων genbase.]{Αποτελέσματα του GMl-ASLCS, του προκατόχου του, GMl-ASLCS$_{\:0}$, και των αλγορίθμων πολυκατηγορικής ταξινόμησης της βιβλιογραφίας οι οποίοι δεν ανήκουν στην οικογένεια των ΜαΣΤ για τις μετρικές της Ακρίβειας, της Ανάκλησης, της Απώλειας Hamming και της Ακριβούς Ορθότητας στο σύνολο δεδομένων genbase.}
\label{table:genbaseEvals}
\begin{tabular}{ccccccccccccccccc}
 & \multicolumn{3}{c}{Ακρίβεια (\%)} & & \multicolumn{3}{c}{Ανάκληση (\%)} & & \multicolumn{3}{c}{Απώλεια Hamming (\%)} & & \multicolumn{3}{c}{Ακριβής Ορθότητα (\%)}
\\ 
\cline{2-4} \cline{6-8} \cline{10-12} \cline{14-16}
& \textsc{PCut} & \textsc{IVal} & \textsc{Best} & & \textsc{PCut} & \textsc{IVal} & \textsc{Best} & & \textsc{PCut} & \textsc{IVal} & \textsc{Best} & & \textsc{PCut} & \textsc{IVal} & \textsc{Best}\\ \hline
GMl-ASLCS$_{\:0}$ & $86.71$ & $87.90$ & $89.45$ & & $91.04$ & $90.86$ & $90.04$ & & $1.07$ & $0.99$ & $\textbf{0.75}$ & & $84.43$ & $84.16$ & $\textbf{88.38}$ \\ 
GMl-ASLCS 		  & $\textbf{97.47}$ & $\textbf{98.63}$ & $\textbf{91.01}$ & & $\textbf{98.81}$ & $98.81$ & $\textbf{97.00}$ & & $\textbf{0.21}$	& $0.12$ & $0.88$	& & $\textbf{94.72}$ & $96.99$	& $83.68$ \\
\hline
\hline
BR-J48 			  & - & $98.62$ & - & & - & $\textbf{99.14}$ & - & & -	& $\textbf{0.11}$ & -	& & - & $\textbf{97.13}$	& - \\
RA$k$EL-J48 	  & - & $98.62$ & - & & - & $\textbf{99.14}$ & - & & -	& $\textbf{0.11}$ & -	& & - & $\textbf{97.13}$	& - \\
Ml$k$NN 		  & - & $94.11$ & - & & - & $94.28$ & - & & -	& $0.50$ & -	& & - & $90.64$	& - \\
\hline

\end{tabular} 
\end{center}
\end{table}


\begin{table}
\begin{center}
\caption[Αποτελέσματα στο σύνολο δεδομένων scene.]{Αποτελέσματα του GMl-ASLCS, του προκατόχου του, GMl-ASLCS$_{\:0}$, και των αλγορίθμων πολυκατηγορικής ταξινόμησης της βιβλιογραφίας οι οποίοι δεν ανήκουν στην οικογένεια των ΜαΣΤ για τις μετρικές της Ακρίβειας, της Ανάκλησης, της Απώλειας Hamming και της Ακριβούς Ορθότητας στο σύνολο δεδομένων scene.}
\label{table:sceneEvals}
\begin{tabular}{ccccccccccccccccc}
 & \multicolumn{3}{c}{Ακρίβεια (\%)} & & \multicolumn{3}{c}{Ανάκληση (\%)} & & \multicolumn{3}{c}{Απώλεια Hamming (\%)} & & \multicolumn{3}{c}{Ακριβής Ορθότητα (\%)}
\\ 
\cline{2-4} \cline{6-8} \cline{10-12} \cline{14-16}
& \textsc{PCut} & \textsc{IVal} & \textsc{Best} & & \textsc{PCut} & \textsc{IVal} & \textsc{Best} & & \textsc{PCut} & \textsc{IVal} & \textsc{Best} & & \textsc{PCut} & \textsc{IVal} & \textsc{Best}\\ \hline
GMl-ASLCS$_{\:0}$ & $41.14$ & $42.39$ & $44.88$ & & $44.98$ & $51.84$ & $45.90$ & & $19.30$ & $21.46$ & $17.66$ & & $34.44$ & $30.27$ & $41.38$ \\ 
GMl-ASLCS 		  & $\textbf{62.82}$ & $63.15$ & $\textbf{52.59}$ & & $\textbf{69.23}$ & $70.23$ & $\textbf{53.30}$ & & $\textbf{12.04}$	& $12.11$ & $\textbf{15.82}$	& & $\textbf{53.43}$ & $53.18$	& $\textbf{48.83}$ \\
\hline
\hline
BR-J48 			  & - & $51.34$ & - & & - & $61.12$ & - & & -	& $13.89$ & -	& & - & $40.13$	& - \\
RA$k$EL-J48 	  & - & $57.76$ & - & & - & $62.17$ & - & & -	& $11.50$ & -	& & - & $50.75$	& - \\
Ml$k$NN 		  & - & $\textbf{66.14}$ & - & & - & $\textbf{77.24}$ & - & & -	& $\textbf{9.53}$  & -	& & - & $\textbf{60.12}$	& - \\
\hline

\end{tabular} 
\end{center}
\end{table}
\end{landscape}


\begin{landscape}
\begin{table}
\begin{center}
\caption[Αποτελέσματα στο σύνολο δεδομένων medical.]{Αποτελέσματα του GMl-ASLCS, του προκατόχου του, GMl-ASLCS$_{\:0}$, και των αλγορίθμων πολυκατηγορικής ταξινόμησης της βιβλιογραφίας οι οποίοι δεν ανήκουν στην οικογένεια των ΜαΣΤ για τις μετρικές της Ακρίβειας, της Ανάκλησης, της Απώλειας Hamming και της Ακριβούς Ορθότητας στο σύνολο δεδομένων medical.}
\label{table:medicalEvals}
\begin{tabular}{ccccccccccccccccc}
 & \multicolumn{3}{c}{Ακρίβεια (\%)} & & \multicolumn{3}{c}{Ανάκληση (\%)} & & \multicolumn{3}{c}{Απώλεια Hamming (\%)} & & \multicolumn{3}{c}{Ακριβής Ορθότητα (\%)}
\\ 
\cline{2-4} \cline{6-8} \cline{10-12} \cline{14-16}
& \textsc{PCut} & \textsc{IVal} & \textsc{Best} & & \textsc{PCut} & \textsc{IVal} & \textsc{Best} & & \textsc{PCut} & \textsc{IVal} & \textsc{Best} & & \textsc{PCut} & \textsc{IVal} & \textsc{Best}\\ \hline
GMl-ASLCS$_{\:0}$ & $39.49$ & $40.14$ & $\textbf{44.92}$ & & $46.20$ & $48.30$ & $47.55$ & & $2.98$ & $3.10$ & $\textbf{2.78}$ & & $28.53$ & $27.93$ & $\textbf{36.34}$ \\ 
GMl-ASLCS 		  & $\textbf{52.34}$ & $51.58$ & $38.73$ & & $\textbf{60.03}$ & $56.80$ & $\textbf{49.72}$ & & $\textbf{2.06}$	& $1.95$ & $3.20$	& & $\textbf{39.84}$ & $41.40$	& $22.79$ \\
\hline
\hline
BR-J48 			  & - & $\textbf{74.26}$ & - & & - & $\textbf{79.41}$ & - & & -	& $\textbf{1.06}$ & -	& & - & $\textbf{65.12}$	& - \\
RA$k$EL-J48 	  & - & $72.84$ & - & & - & $78.22$ & - & & -	& $1.13$ & -	& & - & $64.03$	& - \\
Ml$k$NN 		  & - & $41.77$ & - & & - & $43.31$ & - & & -	& $1.88$ & -	& & - & $35.19$	& - \\
\hline

\end{tabular} 
\end{center}
\end{table}


\begin{table}
\begin{center}
\caption[Αποτελέσματα στο σύνολο δεδομένων enron.]{Αποτελέσματα του GMl-ASLCS, του προκατόχου του, GMl-ASLCS$_{\:0}$, και των αλγορίθμων πολυκατηγορικής ταξινόμησης της βιβλιογραφίας οι οποίοι δεν ανήκουν στην οικογένεια των ΜαΣΤ για τις μετρικές της Ακρίβειας, της Ανάκλησης, της Απώλειας Hamming και της Ακριβούς Ορθότητας στο σύνολο δεδομένων enron.}
\label{table:enronEvals}
\begin{tabular}{ccccccccccccccccc}
				& \multicolumn{3}{c}{Ακρίβεια (\%)} & & \multicolumn{3}{c}{Ανάκληση (\%)} & & \multicolumn{3}{c}{Απώλεια Hamming (\%)} & & \multicolumn{3}{c}{Ακριβής Ορθότητα (\%)}
\\ 
\cline{2-4} \cline{6-8} \cline{10-12} \cline{14-16}
& \textsc{PCut} & \textsc{IVal} & \textsc{Best} & & \textsc{PCut} & \textsc{IVal} & \textsc{Best} & & \textsc{PCut} & \textsc{IVal} & \textsc{Best} & & \textsc{PCut} & \textsc{IVal} & \textsc{Best}\\ \hline
GMl-ASLCS$_{\:0}$ & $38.93$ & $39.40$ & $25.04$ & & $53.67$ & $49.81$ & $29.81$ & & $6.26$ & $6.00$ & $\textbf{6.77}$ & & $\textbf{7.43}$ & $\textbf{11.92}$  & $\textbf{9.15}$ \\ 
GMl-ASLCS 		  & $\textbf{40.36}$ & $40.35$ & $\textbf{27.33}$ & & $\textbf{54.34}$ & $\textbf{54.50}$ & $\textbf{43.74}$ & & $\textbf{5.88}$	& $5.88$ & $8.10$	& & $7.08$ & $6.91$	& $2.59$ \\
\hline
\hline
BR-J48 			  & - & $36.71$ & - & & - & $44.81$ & - & & -	& $5.40$ & -	& & - & $8.64$	& - \\
RA$k$EL-J48 	  & - & $\textbf{41.04}$ & - & & - & $49.46$ & - & & -	& $\textbf{5.09}$ & -	& & - & $10.71$	& - \\
Ml$k$NN 		  & - & $31.85$ & - & & - & $35.80$ & - & & -	& $5.14$ & -	& & - & $6.22$	& - \\
\hline

\end{tabular} 
\end{center}
\end{table}
\end{landscape}


\section{Συγκριτική Ανάλυση Αποτελεσμάτων}
Τα αποτελέσματα που προέκυψαν από την πειραματική διαδικασία σχολιάζονται στις επόμενες παραγράφους.

\subsection{Σύγκριση του GMl-ASLCS με τον προκάτοχό του}
Ο GMl-ASLCS εμφανίζει καλύτερα αποτελέσματα\footnote{Για τις μετρικές της Ακρίβειας, της Ανάκλησης, της Ακριβούς Ορθότητας και της Μέσης Κάλυψης, μεγαλύτερες τιμές είναι καλύτερες. Για την Απώλεια Hamming ισχύει το αντίθετο.} ως προς όλες τις μετρικές σε σχέση με τον GMl-ASLCS, για όλα τα σύνολα δεδομένων, εκτός από τη μετρική της ακρίβειας για το σύνολο δεδομένων medical για τη στρατηγική συμπερασμού best, για τη μετρική της Απώλειας Hamming στα σύνολα δεδομένων yeast, genbase, medical και enron για τη στρατηγική συμπερασμού best και της Ακριβούς Ορθότητας για τα σύνολα δεδομένων genbase, medical και enron για όλες τις στρατηγικές συμπερασμού. Όσον αφορά στη βασική μετρική αξιολόγησης, την ακρίβεια, η προσέγγισή μας τη βελτιώνει κατά $20.82\%$ στο σύνολο music, $13.54\%$ στο σύνολο yeast, $12.21\%$ στο genbase,  
$48.97\%$ στο scene, $28.50\%$ στο medical και κατά $2.41\%$ στο σύνολο enron, με βάση τη μέθοδο επιλογής κατωφλίου IVal.

Στους Πίνακες \ref{table:gmlaslcs0coverage} και \ref{table:gmlaslcsCoverage} παρουσιάζονται, για κάθε σύνολο δεδομένων, το ποσοστό μέσης κάλυψης δειγμάτων (coverage $\%$), το άνω όριο του αριθμού των μικρο-κανόνων του πληθυσμού $\abs{P}$, ο αριθμός των μικρο-κανόνων του τελικού μοντέλου των δύο ΜαΣΤ $\abs{P}_{final}$ και το ποσοστό της απόκλισης του τελικού αριθμού των μικρο-κανόνων του πληθυσμού από το μέγιστο (population lost $\%$). Σε αυτούς τους πίνακες αυτό που πρέπει να παρατηρήσουμε, κατ' αρχάς, είναι η αύξηση του ποσοστού της μέσης κάλυψης δειγμάτων των κανόνων του GMl-ASLCS σε σχέση με αυτό του GMl-ASLCS$_{\:0}$. Αυτή οδηγεί σε απόλυτη αύξηση σε κάλυψη $8$ δειγμάτων για το σύνολο music, $10$ για το yeast, $5.4$ για το genbase, $6.7$ για το scene και $2.89$ δειγμάτων για το σύνολο enron. Αν και στο σύνολο medical παρατηρείται πτώση του ποσοστού κάλυψης, αυτή είναι πλασματική: λόγω της υπερ-παραγωγής κανόνων μηδενικής κάλυψης, η ποιότητα των χρήσιμων κανόνων του πληθυσμού έχει συμπιεστεί σε τέτοιο βαθμό ώστε ο GMl-ASLCS$_{\:0}$ εξελίσσει μόνο πολύ γενικούς κανόνες, περισσότερο δηλαδή από όσο θα έπρεπε ώστε να μην διακινδυνεύει την ακρίβειά του. Το φαινόμενο παραγωγής κανόνων μηδενικής κάλυψης στον GMl-ASLCS$_{\:0}$ έχει τέτοια ένταση που το τελικό σύνολο κανόνων αποτελείται μόλις από το $14.69\%$ του αριθμού των κανόνων που θα έπρεπε να συγκρατεί ο πληθυσμός. 

Η στήλη $\abs{P}_{final}$ στον Πίνακα \ref{table:gmlaslcs0coverage}, σε αντιπαραβολή με τα στοιχεία της στήλης $\abs{P}$, χρησιμοποιείται για να παρουσιάσει το αποτέλεσμα της συγκράτησης στον πληθυσμό κανόνων μηδενικής κάλυψης στον GMl-ASLCS$_{\:0}$, όσον αφορά στα πραγματικά σύνολα δεδομένων, δεδομένου πως στην περίοδο ενημέρωσης $0.1 \cdot \abs{I}$ επαναλήψεων που ακολουθεί το ουσιώδες μέρος της διαδικασίας εκπαίδευσης, η μόνη λειτουργία αφαίρεσης κανόνων από τον πληθυσμό είναι αυτή της αφαίρεσης κανόνων λόγω της διαπίστωσης πως δεν καλύπτουν κανένα δείγμα του συνόλου δεδομένων.

Στον Πίνακα \ref{table:gmlaslcsCoverage} η στήλη population lost $\%$ και, κατά συνέπεια, και η στήλη $\abs{P}_{final}$, χρησιμοποιούνται για να προσδώσουν μία εικόνα του μεγέθους του αριθμού των κανόνων που διαγράφονται από τη λειτουργία διαγραφής κανόνων από τα Match Sets, κατά το διάστημα ενημέρωσης στον GMl-ASLCS. Και στα δύο ΜαΣΤ, η λειτουργία διαγραφής κανόνων μέσω επιλογής ρουλέτας είναι απενεργοποιημένη λόγω της απενεργοποίησης του Γενετικού Αλγορίθμου.

Εν κατακλείδι, ο GMl-ASLCS επιτυγχάνει τους δύο στόχους που είχαμε θέσει: α) τη βελτίωση της ακρίβειάς του, που αποδεικνύεται από τις επιδόσεις του στα έξι πραγματικά σύνολα πολυκατηγορικών δεδομένων που δοκιμάσαμε, και β) την αύξηση των δειγμάτων που καλύπτουν οι κανόνες του πληθυσμού του, μέσω της λειτουργίας διαγραφής κανόνων από τα Match Sets. Συγκεκριμένα, όσον αφορά στην αύξηση της μέσης κάλυψης των κανόνων, φαίνεται ότι αυτή δεν ακυρώνεται και ότι δε δημιουργείται κάποια επιπρόσθετη πίεση προς τη συνολικότερη ειδίκευση των κανόνων, μέσω της τροποποίησης συνιστωσών ή λειτουργιών.



\begin{table}[!h]
\begin{center}
\caption{Μέση Κάλυψη δειγμάτων στα πειράματα του GMl-ASLCS$_{\:0}$.}
\label{table:gmlaslcs0coverage}
    \begin{tabular}{l|crrc}
    \hline \\ [-2ex]
    dataset & coverage $\%$       & $\abs{P}$ & $\abs{P}_{final}$ & population lost $\%$ \\
    \hline \\ [-2ex]
    music   & $0.2827$  & $8000$    & $6618$            & $17.23$           	\\
    yeast   & $0.1361$  & $14000$   & $6673$            & $52.33$            	\\
    genbase & $2.1677$  & $6000$    & $5023$            & $16.28$            	\\
    scene   & $0.1628$  & $10000$   & $4043$            & $59.57$            	\\
    medical & $4.7601$  & $8000$    & $1175$            & $85.31$            	\\
    enron   & $0.3545$  & $10000$   & $6133$            & $38.67$            	\\ 
    \hline
    \end{tabular}
\end{center}
\end{table}



\begin{table}[!h]
\begin{center}
    \caption{Μέση Κάλυψη δειγμάτων στα πειράματα του GMl-ASLCS.}
	\label{table:gmlaslcsCoverage}
    \begin{tabular}{l|crrc}
    \hline \\ [-2ex]
    dataset & coverage $\%$     & $\abs{P}$ & $\abs{P}_{final}$ & population lost $\%$ \\
    \hline \\ [-2ex]
    music   & $1.7896$       	& $5000$    & $4872$            & $2.56$               \\
    yeast   & $0.4733$        	& $18000$   & $17721$           & $1.55$               \\
    genbase & $3.0724$        	& $12000$   & $10980$           & $8.50$                \\
    scene   & $0.4411$ 		  	& $9000$    & $8857$            & $1.59$               \\
    medical & $3.2497$  		& $2500$    & $2498$            & $0.08$               \\
    enron   & $0.5242$ 			& $25000$   & $22841$           & $8.64$               \\ 
    \hline
    \end{tabular}
\end{center}
\end{table}



\subsection{Σύγκριση των Αλγορίθμων με βάση την Ακρίβεια}
\label{subsec:accBasedComparison}
Ο Πίνακας \ref{table:accuracyBasedComparison} συνοψίζει τα ποσοστά της ακριβείας των υπό μελέτη αλγορίθμων για όλα τα χρησιμοποιούμενα σύνολα δεδομένων, για τη στρατηγική συμπερασμού IVal. Μαζί με τις μετρικές επίδοσης του καθενός, αναφέρουμε και τις αντίστοιχες κατατάξεις ανά σύνολο δεδομένων ως εκθέτες στην τιμή της μετρικής, τη μέση κατάταξη κάθε αλγορίθμου όπως αυτή προκύπτει από την εφαρμογή του τεστ Friedman στη στήλη με τίτλο "Κατάταξη", καθώς και την απόλυτη θέση των αλγορίθμων στην τελική κατάταξη ως εκθέτη της τιμής της μέσης κατάταξης.


\begin{table}[!h]
\begin{center}
    \caption[Σύγκριση αλγορίθμων πολυκατηγορικής ταξινόμησης με βάση την ακρίβεια, με μέθοδο ταξινόμησης IVal, στο σύνολο των υπό μελέτη προβλημάτων ταξινόμησης.] {Σύγκριση αλγορίθμων πολυκατηγορικής ταξινόμησης με βάση την ακρίβεια, με στρατηγική συμπερασμού IVal, στο σύνολο των υπό μελέτη προβλημάτων ταξινόμησης. Οι εκθέτες αναφέρονται στην κατάταξη του κάθε αλγορίθμου ανά σύνολο δεδομένων, σύμφωνα με το στατιστικό τεστ Friedman. Η στήλη με τίτλο "Κατάταξη" περιέχει τη συνολική κατάταξη του αλγορίθμου στην αντίστοιχη γραμμή, ενώ ο εκθέτης σημαίνει τη θέση του στην (απόλυτη) τελική κατάταξη.}
	\label{table:accuracyBasedComparison}
    \begin{tabular}{l|cccccc|c}
    \hline \\ [-2ex] 
    Αλγόριθμοι        & music     & yeast     & genbase       & scene     & medical   & enron     & Κατάταξη \\
    \hline \\ [-2ex] 
    GMl-ASLCS$_{\:0}$ & $50.05^4$ & $45.51^4$ & $87.90^5$     & $42.39^5$ & $40.14^5$ & $39.40^3$ & $4.33^5$      \\
    GMl-ASLCS         & $\textbf{60.47}^1$ & $\textbf{51.67}^1$ & $\textbf{98.63}^1$  & $63.15^2$ & $51.58^3$ & $40.35^2$ & $\textbf{1.67}^1$  \\
    BR-J48            & $46.23^5$ & $43.95^5$ & $98.62^{2.5}$ & $51.34^4$ & $\textbf{74.26}^1$ & $36.71^4$ & $3.58^4$    \\
    RA$k$EL-J48       & $50.91^3$ & $48.74^3$ & $98.62^{2.5}$ & $57.76^3$ & $72.84^2$ & $\textbf{41.04}^1$ & $2.42^2$      \\
    Ml$k$NN           & $53.26^2$ & $51.62^2$ & $94.11^4$     & $\textbf{66.14}^1$ & $41.77^4$ & $31.84^5$ & $3.00^3$        
    \\ \hline
    \end{tabular}
\end{center}
\end{table}


Σύμφωνα με τη μέση κατάταξη των αλγορίθμων, ο GMl-ASLCS κατατάσσεται πρώτος, ενώ αμέσως μετά ακολουθούν οι state-of-the-art αλγόριθμοι RA$k$EL-J48, Ml$k$NN, και BR-J48. Στην τελευταία θέση βρίσκεται ο προκάτοχος του GMl-ASLCS, ο GMl-ASLCS$_{\:0}$.

Για τη διερεύνηση της στατιστικής σημαντικότητας των μετρούμενων διαφορών στην κατάταξη των αλγόριθμων, πραγματοποιήθηκε το μη παραμετρικό στατιστικό τεστ Friedman \cite{Friedman1940}, με παραμέτρους $k=5$ και $N=6$, το οποίο \emph{απέρριψε τη μηδενική υπόθεση (null hypothesis-H$_{0}$)}, δηλαδή την υπόθεση ότι όλοι οι αλγόριθμοι έχουν ισοδύναμη απόδοση, για επίπεδο εμπιστοσύνης $\alpha = 0.05$. Για τον εντοπισμό των μεθόδων μεταξύ των οποίων υπάρχει στατιστικά σημαντική διαφορά απόδοσης, εκτελέστηκε η post-hoc δοκιμή Nemenyi \cite{nemenyi63} σε επίπεδο εμπιστοσύνης $\alpha=0.05$, η οποία αποκάλυψε ότι:

\begin{itemize}
\item \emph{δεν υπάρχει στατιστικά σημαντική διαφορά στην απόδοση μεταξύ του GMl-ASLCS και των αντιπάλων του αλγορίθμων που δεν ανήκουν στην οικογένεια των ΜαΣΤ (RA$k$EL-J48, Ml$k$NN, BR-J48)}
\item \emph{υπάρχει στατιστικά σημαντική διαφορά στην απόδοση του GMl-ASLCS σε σχέση με τον GMl-ASLCS$_{\:0}$}.
\end{itemize}



\subsection{Σύγκριση των Αλγορίθμων με βάση την Ακριβή Ορθότητα}
\label{subsec:exBasedComparison}

Ο Πίνακας \ref{table:exactMatchBasedComparison} συνοψίζει τα ποσοστά της ακριβούς ορθότητας των υπό μελέτη αλγορίθμων για όλα τα χρησιμοποιούμενα σύνολα δεδομένων, για τη στρατηγική συμπερασμού IVal. Οι τιμές αντιστοιχούν σε αυτές των πειραμάτων που διενεργήθηκαν και απέδωσαν τις τιμές ακρίβειας της προηγούμενης παραγράφου. Μαζί με τις μετρικές επίδοσης του καθενός, αναφέρουμε και τις αντίστοιχες κατατάξεις ανά σύνολο δεδομένων ως εκθέτες στην τιμή της μετρικής, τη μέση κατάταξη κάθε αλγορίθμου όπως αυτή προκύπτει από την εφαρμογή του τεστ Friedman στη στήλη με τίτλο "Κατάταξη", καθώς και την απόλυτη θέση των αλγορίθμων στην τελική κατάταξη, ως εκθέτη της τιμής της μέσης κατάταξης.

\begin{table}[!h]
\begin{center}
    \caption[Σύγκριση αλγορίθμων πολυκατηγορικής ταξινόμησης με βάση την Ακριβή Ορθότητα, με μέθοδο ταξινόμησης IVal, στο σύνολο των υπό μελέτη προβλημάτων ταξινόμησης.] {Σύγκριση αλγορίθμων πολυκατηγορικής ταξινόμησης με βάση την ακριβή ορθότητα, με στρατηγική συμπερασμού IVal, στο σύνολο των υπό μελέτη προβλημάτων ταξινόμησης. Οι εκθέτες αναφέρονται στην κατάταξη του κάθε αλγορίθμου ανά σύνολο δεδομένων, σύμφωνα με το στατιστικό τεστ Friedman. Η στήλη με τίτλο "Κατάταξη" περιέχει τη συνολική κατάταξη του αλγορίθμου στην αντίστοιχη γραμμή, ενώ ο εκθέτης σημαίνει τη θέση του στην (απόλυτη) τελική κατάταξη.}
	\label{table:exactMatchBasedComparison}
    \begin{tabular}{l|llllll|c}
        \hline \\ [-2ex] 
    Αλγόριθμοι        & music     & yeast     & genbase       & scene     & medical   & enron  & Κατάταξη \\
    \hline \\ [-2ex] 
    GMl-ASLCS$_{\:0}$ & $23.11^2$ & $5.01^5$  & $84.16^5$     & $30.27^5$ & $27.93^5$ & $\textbf{11.92}^1$ & $3.83^5$     \\
    GMl-ASLCS         & $\textbf{34.39}^1$ & $11.36^3$ & $96.99^3$     & $53.18^2$ & $41.40^3$ & $6.91^4$  & $2.67^2$   \\
    BR-J48            & $6.83^5$  & $6.83^4$  & $\textbf{97.13}^{1.5}$ & $40.13^4$ & $\textbf{65.12}^1$ & $8.64^3$  & $3.08^4$   \\
    RA$k$EL-J48       & $11.71^4$ & $11.71^2$ & $\textbf{97.13}^{1.5}$ & $50.75^3$ & $64.03^2$ & $10.71^2$ & $\textbf{2.42}^1$    \\
    Ml$k$NN           & $18.74^3$ & $\textbf{18.74}^1$ & $90.64^4$     & $\textbf{60.12}^1$ & $35.19^4$ & $6.22^5$  & $3.00^3$     \\ \hline
    \end{tabular}
\end{center}
\end{table}

Σύμφωνα με τη μέση κατάταξη των αλγορίθμων, ο RA$k$EL-J48 κατατάσσεται πρώτος, ενώ ακολουθεί ο GMl-ASLCS στη δεύτερη θέση, με τους Ml$k$NN και BR-J48 να ακολουθούν στην τρίτη και τέταρτη θέση αντίστοιχα. Ο GMl-ASLCS$_{\:0}$ κατατάσσεται και εδώ στην τελευταία θέση.

Για τη διερεύνηση της στατιστικής σημαντικότητας των μετρούμενων διαφορών στην κατάταξη των αλγορίθμων για τη μετρική της ακριβούς ορθότητας, πραγματοποιήθηκε το μη παραμετρικό στατιστικό τεστ Friedman, με παραμέτρους $k=5$ και $N=6$, όπως και στην προηγούμενη περίπτωση, το οποίο \emph{δεν απέρριψε τη μηδενική υπόθεση}, για όλα τα επίπεδα εμπιστοσύνης. Συνεπώς, αυτό που συμπεραίνουμε είναι ότι δεν υπάρχει στατιστικά σημαντική διαφορά στις επιδόσεις των πέντε αλγορίθμων, όσον αφορά στη μετρική της ακριβούς ορθότητας.

\section{Επίδραση των Διαφοροποιήσεων στην Επίδοση του GMl-ASLCS}
Σε αυτή την Ενότητα διερευνούμε δύο ειδών επιδράσεις:
\begin{enumerate}
\item αυτές των βασικών λειτουργιών που αντικαταστήσαμε ή εισηγάγαμε στον ορισμό του GMl-ASLCS ως πρωτότυπες λειτουργίες, ώστε να εξακριβωθεί η διαφορά που αυτές επιφέρουν στις επιδόσεις του GMl-ASLCS (Παρ. \ref{subsec:gmlaslcsCrossover} και \ref{subsec:controlInM}), και
\item αυτές των τροποποιήσεων που προτείναμε στην Εν. \ref{sec:alterations}.
\end{enumerate}

Στην πρώτη υπό μελέτη ομάδα επιδράσεων ανήκει η εξέταση των επιδόσεων του GMl-ASLCS α) με χρήση του Γενετικού Αλγορίθμου με Διασταύρωση Ενός Σημείου και β) απουσία της διαγραφής κανόνων από τα Match Sets, η οποία αναμένεται ότι θα μειώσει τα επίπεδα μέσης κάλυψης δειγμάτων σε αυτά του GMl-ASLCS$_{\:0}$, με απροσδιόριστες όμως τιμές της τελικής ακρίβειας. Θα ονομάσουμε το πρώτο ΜαΣΤ GMl-ASLCS$_{sp\chi}$ και το δεύτερο GMl-ASLCS$_{!M}$. 

Στην περίπτωση του GMl-ASLCS$_{!M}$, λόγω της περαιτέρω διαγραφής κανόνων, υπάρχουν δύο δυνατότητες: α) να θέσουμε το μέγεθος του πληθυσμού ίσο με αυτό των βασικών πειραμάτων (GMl-ASLCS$_{!Ms}$) ή β) να αυξήσουμε το μέγεθος του πληθυσμού ανάλογα με τον αριθμό των κανόνων που διαγράφηκαν από τη λειτουργία διαγραφής κανόνων από τα Match Sets (GMl-ASLCS$_{!Ma}$). Σε αυτή την περίπτωση το μέγεθος του πληθυσμού υπολογίζεται από τη σχέση 
\begin{equation}
\label{eq:populationAugmentationFormula}
\abs{P}=(\frac{M}{M+d+\abs{P_{0}}} + 1) \cdot \abs{P_{0}}
\end{equation}
\\
όπου $M$ ο αριθμός των κανόνων που διαγράφηκαν από τη λειτουργία διαγραφής κανόνων από τα Match Sets, $d$ ο αριθμός των κανόνων που διαγράφηκαν από την τυπική λειτουργία διαγραφής, με επιλογή κανόνων μέσω επιλογής ρουλέτας, και $\abs{P_{0}}$ το μέγεθος του πληθυσμού που υπαγορεύεται από τη στήλη $\abs{P}$ του Πίνακα \ref{table:realDatasetsParams}. Το άθροισμα $S=M+d+\abs{P_{0}}$ είναι ο συνολικός αριθμός κανόνων που παρήχθησαν από το ΜαΣΤ στο σύνολο της εκπαίδευσης, συνεπώς ο λόγος $M/S$ εκφράζει το ποσοστό των κανόνων που διαγράφηκαν από τη λειτουργία διαγραφής κανόνων στα Match Sets. Κατά το ποσοστό αυτό αυξάνουμε το μέγεθος του πληθυσμού, ώστε εν τέλει να έχει παραχθεί ο ίδιος αριθμός κανόνων, χωρίς να έχει αφαιρεθεί κάποιος από τα Match Sets.

Στη δεύτερη υπό μελέτη ομάδα επιδράσεων ανήκει α) η τροποποίηση της μεθόδου υπολογισμού της καταλληλότητας στη συνιστώσα ενίσχυσης, η οποία περιγράφηκε στην Παρ. \ref{subsec:qfitness} (GMl-ASLCS$_{f}$),  β) αυτή της μεθόδου υπολογισμού ανάθεσης πιθανότητας διαγραφής κανόνων, η οποία παρουσιάστηκε στην Παρ. \ref{subsec:ucsDeletion} (GMl-ASLCS$_{d}$), γ) η μεταβολή των μεταβλητών $\omega, \phi$ από $(\omega, \phi) \equiv (0.9,1)$ σε $(\omega, \phi) \equiv (0,0)$ (GMl-ASLCS$_{!\#}$) και δ) ο GMl-ASLCS που κάνει χρήση της λειτουργίας ομαδοποίησης για την αρχικοποίηση του πληθυσμού (Παρ. \ref{subsec:gmlaslcsClustering}). Στην τελευταία περίπτωση μελετήθηκαν τρεις μέθοδοι ομαδοποίησης: 
\begin{enumerate}
\item Η ομαδοποίηση με παραμέτρους $\gamma=0.01$ και $Pcl_{\#A}=Pcl_{\#L}=0$ (GMl-ASLCS$_{Cl.sp}$), ώστε σε προβλήματα με ανισορροπία συνδυασμών ετικετών, που απαιτούν την εύρεση συγκεκριμένων, ειδικών κανόνων, αυτοί να τους παρασχεθούν από τη λειτουργία ομαδοποίησης. Η τιμή $\gamma=0.01$ θεωρείται, και είναι, εξαιρετικά μικρή σε σχέση με τις συνήθεις τιμές της παραμέτρου $\gamma$. Η χρήση μίας μικρής τιμής για την παράμετρο $\gamma$, θα έχει ως αποτέλεσμα το σχηματισμό ενός μικρού αριθμού συστάδων για κάθε διακριτό συνδυασμό ετικετών του συνόλου δεδομένων. Πιο συγκεκριμένα, ο αριθμός των συστάδων θα είναι ένα, για κάθε partition με λιγότερα από $100$ δείγματα. Σε συνδυασμό με μηδενικές πιθανότητες γενίκευσης γνωρισμάτων και ετικετών για τους κανόνες που θα δημιουργηθούν μέσω ομαδοποίησης, για τα σύνολα δεδομένων που μελετάμε σε αυτή την εργασία, αναμένουμε την εύρεση ειδικών κανόνων, τόσο στο τμήμα συνθήκης όσο και στο τμήμα απόφασης, που θα περιγράφουν με περιεκτικό τρόπο το τμήμα του χώρου αναζήτησης που ενδεχομένως δε θα μπορούσε να εξερευνήσει ο GMl-ASLCS λόγω της ανισορροπίας συνδυασμού ετικετών.

\item Η ομαδοποίηση με παραμέτρους $\gamma=0.2$ και $Pcl_{\#A}=Pcl_{\#L}=0$, (GMl-ASLCS$_{Cl.lsp}$) για την αρχικοποίηση με περισσότερους, ειδικούς κανόνες που συμπυκνώνουν τη γνώση περισσότερων τμημάτων του χώρου αναζήτησης.

\item Η ομαδοποίηση με παράμετρο $\gamma=0.2$ και πιθανότητες γενίκευσης αυτές που χρησιμοποιήθηκαν στα βασικά πειράματα του GMl-ASLCS (GMl-ASLCS$_{Cl.lge}$), σύμφωνα με τον Πίνακα \ref{table:realDatasetsParams}, ώστε η αρχικοποίηση των κανόνων να γίνει σε μεγαλύτερο βαθμό από τη λειτουργία ομαδοποίησης και σε μικρότερο από τη λειτουργία κάλυψης. Έτσι, οι αρχικοί κανόνες θα βασίζονται σε μία καλύτερη περίληψη του συνόλου δεδομένων, αντί να εξαρτώνται από τη σειρά εισαγωγής των δειγμάτων του και την τυχαιότητα που εισάγει το Τμήμα Κάλυψης.
\end{enumerate}

Συνοπτικά, η ονοματοδοσία των τροποποιήσεων παρουσιάζεται στον Πίνακα \ref{tbl:variationsNames}.


\begin{table}[!h]
\begin{center}
    \caption{Ονοματοδοσία των τροποποιήσεων του GMl-ASLCS.}
	\label{tbl:variationsNames}
    \begin{tabular}{|l | p{12cm}|}
    \hline \\ [-2ex] 
    Αλγόριθμοι           & Αναφερόμενη Τροποποίηση του GMl-ASLCS     \\ \hline \\ [-2ex]
    GMl-ASLCS$_{sp\chi}$  & Αντικατάσταση του τελεστή Διασταύρωσης Δύο Τμημάτων από τον τελεστή Διασταύρωσης Ενός Σημείου.\\ \hline
    GMl-ASLCS$_{!Ms}$    & Αφαίρεση της λειτουργίας διαγραφής κανόνων από τα Match Sets, κρατώντας το μέγεθος του πληθυσμού ίδιο με αυτό που χρησιμοποιήθηκε στα πειράματα της Παρ. \ref{subsec:realDatasetsExperiments} (Πίνακας \ref{table:realDatasetsParams}).\\ \hline
    GMl-ASLCS$_{!Ma}$    & Αφαίρεση της λειτουργίας διαγραφής κανόνων από τα Match Sets, αυξάνοντας το μέγεθος του πληθυσμού σύμφωνα με την Εξ. \ref{eq:populationAugmentationFormula}, σε αντιστοιχία με το μέγεθος του πληθυσμού (Πίνακας \ref{table:realDatasetsParams}) για το κάθε σύνολο δεδομένων.\\ \hline  \hline \\ [-2ex]
    GMl-ASLCS$_{f}$  &  Αντικατάσταση του τρόπου υπολογισμού της καταλληλότητας ενός κανόνα, σύμφωνα με την Παρ. \ref{subsec:qfitness}\\ \hline
    GMl-ASLCS$_{d}$  &   Αντικατάσταση της μεθοδολογίας υπολογισμού πιθανοτήτων διαγραφής, σύμφωνα με την Παρ. \ref{subsec:ucsDeletion}. \\ \hline
    GMl-ASLCS$_{!\#}$  &  Αντικατάσταση των παραμέτρων $\phi$, $\omega$ με τις τιμές $(\phi, \omega) \equiv (0,0)$ (Παρ. \ref{subsec:phiomega}). \\ \hline
    GMl-ASLCS$_{Cl.sp}$  & Αρχικοποίηση του πληθυσμού μέσω ομαδοποίησης, με παραμέτρους $(\gamma, Pcl_{\#A}, Pcl_{\#L}) \equiv (0.01,0,0)$.\\ \hline
    GMl-ASLCS$_{Cl.lsp}$ & Αρχικοποίηση του πληθυσμού μέσω ομαδοποίησης, με παραμέτρους $(\gamma, Pcl_{\#A}, Pcl_{\#L}) \equiv (0.2,0,0)$.\\ \hline
    GMl-ASLCS$_{Cl.lge}$ & Αρχικοποίηση του πληθυσμού μέσω ομαδοποίησης, με παραμέτρους $\gamma=0.2$ και πιθανότητες γενίκευσης που προσδιορίζονται στον Πίνακα \ref{table:realDatasetsParams}.
    
        \\ \hline
    \end{tabular}
        \end{center}

\end{table}



\subsection{Σύγκριση των Αλγορίθμων με βάση την ακρίβεια}

Ο Πίνακας \ref{table:accuracyBasedComparisonVars} συνοψίζει τα ποσοστά της ακρίβειας των υπό μελέτη αλγορίθμων για όλα τα χρησιμοποιούμενα σύνολα δεδομένων, για τη στρατηγική συμπερασμού IVal. Μαζί με τις μετρικές επίδοσης του καθενός, αναφέρουμε και τις αντίστοιχες κατατάξεις ανά σύνολο δεδομένων ως εκθέτες στην τιμή της μετρικής, τη μέση κατάταξη κάθε αλγορίθμου όπως αυτή προκύπτει από την εφαρμογή του τεστ Friedman στη στήλη με τίτλο “Κατάταξη", καθώς και την απόλυτη θέση των αλγορίθμων στην τελική κατάταξη ως εκθέτη της τιμής της μέσης κατάταξης.

\begin{table}[!h]
\begin{center}
    \caption[Σύγκριση του GMl-ASLCS και των τροποποιημένων αλγορίθμων πολυκατηγορικής ταξινόμησης με βάση την ακρίβεια, με μέθοδο ταξινόμησης IVal, στο σύνολο των υπό μελέτη προβλημάτων ταξινόμησης.] {Σύγκριση του GMl-ASLCS και των τροποποιημένων αλγορίθμων πολυκατηγορικής ταξινόμησης με βάση την ακρίβεια, με στρατηγική συμπερασμού IVal, στο σύνολο των υπό μελέτη προβλημάτων ταξινόμησης. Οι εκθέτες αναφέρονται στην κατάταξη του κάθε αλγορίθμου ανά σύνολο δεδομένων, σύμφωνα με το στατιστικό τεστ Friedman. Η στήλη με τίτλο "Κατάταξη" περιέχει τη συνολική κατάταξη του αλγορίθμου στην αντίστοιχη γραμμή, ενώ ο εκθέτης σημαίνει τη θέση του στην (απόλυτη) τελική κατάταξη.}
	\label{table:accuracyBasedComparisonVars}
    \begin{tabular}{l|cccccc|c}
    \hline \\ [-2ex] 
    Αλγόριθμοι           & music         & yeast        & genbase      & scene   & medical      & enron        & Κατάταξη  \\    \hline \\ [-2ex] 
    GMl-ASLCS            & $60.47^2$     & $\textbf{51.67}^1$    & $\textbf{98.63}^1$    & $63.15^3$ & $\textbf{51.58}^1$    & $\textbf{40.35}^1$    & $1.50^1$        \\ \hline \\ [-2ex]
    GMl-ASLCS$_{sp\chi}$  & $57.30^9$     & $51.26^4$    & $97.85^9$    & $62.07^4$ & $49.65^2$    & $37.51^4$    & $5.33^4$        \\
    GMl-ASLCS$_{!Ms}$    & $54.80^{10}$  & $48.35^9$    & $98.28^4$    & $55.42^{10}$ & $45.68^8$    & $34.04^{10}$ & $8.00^9$          \\
    GMl-ASLCS$_{!Ma}$    & $57.31^8$     & $48.46^8$    & $97.81^{10}$ & $58.67^9$ & $46.71^6$    & $36.33^9$    & $8.33^{10}$         \\ \hline \\ [-2ex]
    GMl-ASLCS$_{f}$      & $58.09^5$     & $50.46^7$    & $98.03^6$    & $61.69^6$ & $46.69^7$    & $36.89^6$    & $6.17^6$        \\
    GMl-ASLCS$_{d}$      & $57.73^{6.5}$ & $47.04^{10}$ & $98.39^3$    & $64.33^2$ & $46.82^5$    & $36.56^7$    & $5.58^5$          \\
    GMl-ASLCS$_{!\#}$    & $59.15^4$     & $51.29^3$    & $98.08^5$    & $\textbf{65.31}^1$ & $47.42^4$    & $38.63^2$    & $3.17^2$         \\
    GMl-ASLCS$_{Cl.sp}$  & $59.91^3$     & $50.76^6$    & $98.02^7$    & $60.08^8$      & $45.23^9$    & $37.16^5$    & $6.33^7$          \\
    GMl-ASLCS$_{Cl.lsp}$ & $\textbf{60.51}^1$     & $51.41^2$    & $98.46^2$    & $61.91^5$ & $48.15^3$    & $36.49^8$    & $3.50^3$          \\
    GMl-ASLCS$_{Cl.lge}$ & $57.73^{6.5}$ & $50.94^5$    & $97.99^8$    & $61.51^7$ & $42.74^{10}$ & $37.92^3$    & $6.58^8$         \\ \hline
    \end{tabular}
        \end{center}

\end{table}


Σύμφωνα με τη μέση κατάταξη των αλγορίθμων, ο GMl-ASLCS κατατάσσεται πρώτος, ενώ αμέσως μετά ακολουθεί η τροποποίηση του με παραμέτρους $(\omega, \phi) \equiv (0,0)$ και τρίτος κατατάσσεται ο GMl-ASLCS που χρησιμοποιεί ομαδοποίηση με παραμέτρους $(\gamma, Pcl_{\#A}, Pcl_{\#L}) \equiv (0.2, 0, 0)$.

Για τη διερεύνηση της στατιστικής σημαντικότητας των μετρούμενων διαφορών στην κατάταξη των αλγόριθμων, πραγματοποιήθηκε το μη παραμετρικό στατιστικό τεστ Friedman, με παραμέτρους $k=10$ και $N=6$, το οποίο \emph{απέρριψε τη μηδενική υπόθεση} για επίπεδο εμπιστοσύνης $\alpha = 0.01$. Για τον εντοπισμό των μεθόδων μεταξύ των οποίων υπάρχει στατιστικά σημαντική διαφορά απόδοσης, εκτελέστηκε η post-hoc δοκιμή Nemenyi. Σε επίπεδο εμπιστοσύνης $\alpha=0.05$, η δοκιμή αποκάλυψε ότι υπάρχει στατιστικά σημαντική διαφορά απόδοσης ανάμεσα α) στον GMl-ASLCS και τους GMl-ASLCS$_{Cl.lge}$, GMl-ASLCS$_{!Ms}$ και GMl-ASLCS$_{!Ma}$, και β) ανάμεσα στον GMl-ASLCS$_{!\#}$ και τον GMl-ASLCS$_{!Ma}$. Σε επίπεδο εμπιστοσύνης $\alpha=0.10$, υπάρχει στατιστικά σημαντική διαφορά ανάμεσα στον GMl-ASLCS και τους GMl-ASLCS$_{f}$, GMl-ASLCS$_{Cl.sp}$, GMl-ASLCS$_{Cl.lge}$, GMl-ASLCS$_{!Ms}$ και GMl-ASLCS$_{!Ma}$. Παράλληλα υπάρχει στατιστικά σημαντική διαφορά ανάμεσα στον GMl-ASLCS$_{!\#}$ και τους GMl-ASLCS$_{!Ms}$ και GMl-ASLCS$_{!Ma}$.




\subsection{Σύγκριση των Αλγορίθμων με βάση την Ακριβή Ορθότητα}

\begin{table}[!h]
\begin{center}
    \caption[Σύγκριση του GMl-ASLCS και των τροποποιημένων αλγορίθμων πολυκατηγορικής ταξινόμησης με βάση την ακριβή ορθότητα, με μέθοδο ταξινόμησης IVal, στο σύνολο των υπό μελέτη προβλημάτων ταξινόμησης.] {Σύγκριση του GMl-ASLCS και των τροποποιημένων αλγορίθμων πολυκατηγορικής ταξινόμησης με βάση την ακριβή ορθότητα, με στρατηγική συμπερασμού IVal, στο σύνολο των υπό μελέτη προβλημάτων ταξινόμησης. Οι εκθέτες αναφέρονται στην κατάταξη του κάθε αλγορίθμου ανά σύνολο δεδομένων, σύμφωνα με το στατιστικό τεστ Friedman. Η στήλη με τίτλο “Κατάταξη" περιέχει τη συνολική κατάταξη του αλγορίθμου στην αντίστοιχη γραμμή, ενώ ο εκθέτης σημαίνει τη θέση των αλγορίθμων στην (απόλυτη) τελική κατάταξη.}
	\label{table:exactMatchBasedComparisonVars}
    \begin{tabular}{l|cccccc|c}
    \hline \\ [-2ex] 
    Αλγόριθμος           & music        & yeast       & genbase       & scene   & medical      & enron        & Κατάταξη   \\   \hline \\ [-2ex] 
    GMl-ASLCS            & $34.39^2$    & $11.36^4$   & $\textbf{96.99}^1$     & $53.18^{4.5}$ & $\textbf{41.40}^1$    & $6.91^{3.5}$ & $2.67^1$          \\ \hline \\ [-2ex]
    GMl-ASLCS$_{sp\chi}$  & $30.35^8$    & $10.93^5$   & $95.48^9$     & $50.42^8$ & $36.59^5$    & $5.35^7$     & $7.00^9$       \\
    GMl-ASLCS$_{!Ms}$    & $29.70^{10}$ & $9.84^9$    & $96.08^4$     & $45.82^{10}$ & $30.85^8$    & $2.76^{10}$  & $8.50^{10}$         \\
    GMl-ASLCS$_{!Ma}$    & $30.39^7$    & $10.59^6$   & $95.33^{10}$  & $47.58^9$ & $37.21^3$    & $7.08^2$     & $6.17^{7.5}$         \\ \hline \\ [-2ex]
    GMl-ASLCS$_{f}$      & $32.08^5$    & $10.46^7$   & $95.62^8$     & $52.76^6$ & $31.94^7$    & $3.63^8$     & $6.83^8$      \\
    GMl-ASLCS$_{d}$      & $30.89^6$    & $6.14^{10}$ & $96.53^3$     & $54.35^2$ & $35.50^6$    & $3.11^9$     & $6.00^6$         \\
    GMl-ASLCS$_{!\#}$    & $32.65^4$    & $\textbf{12.28}^1$   & $95.92^{5.5}$ & $\textbf{56.27}^1$ & $37.67^2$    & $6.91^{3.5}$ & $2.83^2$        \\
    GMl-ASLCS$_{Cl.sp}$  & $33.75^3$    & $10.30^8$   & $95.92^{5.5}$ & $51.00^7$      & $29.77^9$    & $\textbf{7.25}^1$     & $5.58^4$         \\
    GMl-ASLCS$_{Cl.lsp}$ & $\textbf{35.24}^1$    & $11.51^2$   & $96.68^2$     & $53.18^{4.5}$ & $36.90^4$    & $5.87^6$     & $3.25^3$          \\
    GMl-ASLCS$_{Cl.lge}$ & $30.26^9$    & $11.56^3$   & $95.78^7$     & $53.68^3$ & $24.50^{10}$ & $6.74^5$     & $6.17^{7.5}$         \\ \hline
    \end{tabular}
    \end{center}
\end{table}


Σύμφωνα με τη μέση κατάταξη των αλγορίθμων, ο GMl-ASLCS κατατάσσεται πρώτος, ενώ και εδώ αμέσως μετά ακολουθεί η τροποποίηση του με παραμέτρους $(\omega, \phi) \equiv (0,0)$ και τρίτος κατατάσσεται ο GMl-ASLCS που χρησιμοποιεί ομαδοποίηση με παραμέτρους $(\gamma, Pcl_{\#A}, Pcl_{\#L}) \equiv (0.2, 0, 0)$.

Για τη διερεύνηση της στατιστικής σημαντικότητας των μετρούμενων διαφορών στην κατάταξη των αλγόριθμων, πραγματοποιήθηκε το μη παραμετρικό στατιστικό τεστ Friedman, με παραμέτρους $k=10$ και $N=6$, το οποίο \emph{απέρριψε τη μηδενική υπόθεση} για επίπεδο εμπιστοσύνης $\alpha = 0.01$. Για τον εντοπισμό των μεθόδων μεταξύ των οποίων υπάρχει στατιστικά σημαντική διαφορά απόδοσης, εκτελέστηκε η post-hoc δοκιμή Nemenyi, σε επίπεδο εμπιστοσύνης $\alpha=0.05$, η οποία αποκάλυψε ότι υπάρχει στατιστικά σημαντική διαφορά στην απόδοση α) ανάμεσα στον GMl-ASLCS και τον GMl-ASLCS$_{!Ms}$, β) ανάμεσα στον GMl-ASLCS$_{!\#}$ και τον GMl-ASLCS$_{!Ms}$ και γ) ανάμεσα στον GMl-ASLCS$_{Cl.lsp}$ και τον GMl-ASLCS$_{!Ms}$.


\subsection{Σχόλια πάνω στα αποτελέσματα}
\label{subsec:resComm}
Από τα παραπάνω συγκριτικά αποτελέσματα και τη στατιστική ανάλυσή τους συνάγουμε τα εξής συμπεράσματα:
\begin{itemize}
\item Ο GMl-ASLCS που χρησιμοποιεί τις τιμές παραμέτρων $\omega=0$ και $\phi=0$, και ο GMl-ASLCS που κάνει χρήση της ομαδοποίησης με παραμέτρους $\gamma=0.2$, $Pcl_{\#A}=0$ και $Pcl_{\#L}=0$, επιδεικνύουν συνεπή και ακριβή συμπεριφορά, στα πρότυπα των επιδόσεων του θεμελιώδους GMl-ASLCS, κατατασσόμενοι συνολικά στη δεύτερη και τρίτη θέση πίσω από αυτόν, αντίστοιχα, και για τις δύο μετρικές ορθότητας.

\item Ο GMl-ASLCS που χρησιμοποιεί τη Διασταύρωση Ενός Σημείου υπολείπεται σε επιδόσεις σε σχέση με τον GMl-ASLCS που χρησιμοποιεί τη Διασταύρωση Δύο Τμημάτων, κυρίως ως προς τη μετρική της ακριβούς ορθότητας, αλλά δεν εμφανίζει στατιστικά σημαντική διαφορά στις μετρικές της ακρίβειας ως προς αυτόν. Συνεπώς, η Διασταύρωση Δύο Τμημάτων φαίνεται ότι είναι αποτελεσματικότερη και υπερτερεί σε σχέση με τη Διασταύρωση Ενός Σημείου και για αυτό είναι προτιμότερη στην ενσωμάτωσή της στον GMl-ASLCS, αλλά και εν γένει σε πολυκατηγορικά ΜαΣΤ.

\item Η αφαίρεση κανόνων με κριτήρια που τίθενται στα Match Sets είναι κρίσιμης σημασίας, όχι μόνο ως προς τις μετρικές της ακρίβειας και της ακριβούς ορθότητας, για τις οποίες τα δύο τροποποιημένα ΜαΣΤ, GMl-ASLCS$_{!Ms}$ και GMl-ASLCS$_{!Ma}$, εμφανίζουν στατιστικά σημαντικές διαφορές σε σχέση με τον GMl-ASLCS, αλλά και ως προς τη μετρική της μέσης κάλυψης δειγμάτων, όπως φαίνεται στον Πίνακα \ref{table:coverageBasedComparisonVars}. Με άλλα λόγια, η λειτουργία διαγραφής κανόνων από τα Match Sets, στο πλαίσιο λειτουργίας του GMl-ASLCS είναι απολύτως απαραίτητη, τόσο για την αύξηση των επιδόσεων ως προς τις μετρικές της ακρίβειας και της ακριβούς ορθότητας, όσο και για την αύξηση των δειγμάτων που καλύπτουν οι κανόνες του τελικού μοντέλου του GMl-ASLCS.

\item Η αρχικοποίηση του πληθυσμού μέσω ομαδοποίησης απαιτεί προσεκτικό χειρισμό των υπό μεταβολή παραμέτρων. Τα πειράματά μας δείχνουν ότι η καλύτερη συμπεριφορά του GMl-ASLCS επιτυγχάνεται για έναν μικρό ($\gamma=0.2$), αλλά όχι πολύ μικρό ($\gamma=0.01$), αριθμό συστάδων (διαφορά του GMl-ASLCS$_{Cl.lsp}$ με τον GMl-ASLCS$_{Cl.sp}$) και για μικρότερες τιμές πιθανοτήτων γενίκευσης γνωρισμάτων και ετικετών, στη συγκεκριμένη περίπτωση για $(P_{\#A}, P_{\#L}) \equiv (0,0)$ (διαφορά του GMl-ASLCS$_{Cl.lsp}$ με τον GMl-ASLCS$_{Cl.lge}$).

\item Ο GMl-ASLCS με τροποποιημένη μέθοδο υπολογισμού ανάθεσης πιθανοτήτων διαγραφής υπό συνθήκες μπορεί να οδηγήσει τον πληθυσμό των κανόνων σε μεγαλύτερα επίπεδα γενίκευσης (Πίνακας \ref{table:coverageBasedComparisonVars}) από αυτά που απαιτούνται ώστε να υπάρχει ευσταθής ισορροπία ανάμεσα στη γενίκευση και την ακρίβεια του τελικού μοντέλου, υποσκάπτοντας τα επίπεδα ακρίβειας και Ακριβούς Ορθότητας, όπως φαίνεται στα αποτελέσματα του GMl-ASLCS$_{d}$ για το σύνολο yeast.
\end{itemize}


Στον Πίνακα \ref{table:coverageBasedComparisonVars} παρατίθενται ενδεικτικές τιμές της μέσης κάλυψης δειγμάτων (ως ποσοστού επί τοις εκατό) από τον GMl-ASLCS και τις τροποποιήσεις του GMl-ASLCS$_{*}$. Αξιοσημείωτο είναι το γεγονός ότι οι αλγόριθμοι που δεν κάνουν χρήση της λειτουργίας διαγραφής κανόνων από τα Match Sets (GMl-ASLCS$_{!M*}$), εμφανίζουν σημαντική διαφορά, όσον αφορά στα επίπεδα μέσης κάλυψης δειγμάτων, σε σχέση με τους υπόλοιπους αλγορίθμους, οι οποίοι χρησιμοποιούν στο σύνολό τους την παραπάνω λειτουργία.

\begin{table}[!h]
\begin{center}
    \caption{Ενδεικτικές τιμές της μετρικής Μέσης Κάλυψης δειγμάτων (ποσοστό επί τοις εκατό) για τον GMl-ASLCS και τους τροποποιημένους αλγορίθμους πολυκατηγορικής ταξινόμησης GMl-ASLCS$_{*}$, στο σύνολο των υπό μελέτη προβλημάτων ταξινόμησης.}
	\label{table:coverageBasedComparisonVars}
    \begin{tabular}{l|llllll}
        \hline \\ [-2ex] 
    Αλγόριθμος           	& music      	& yeast         & genbase       & scene     & medical       & enron        \\     \hline \\ [-2ex] 
    GMl-ASLCS            	& $1.7896$    	& $0.47329$   	& $3.0724$    	& $0.4411$  & $3.2497$    & $\textbf{0.5242}$          \\ \hline \\ [-2ex]
    GMl-ASLCS$_{sp\chi}$  	& $1.6628$    	& $0.4664$    	& $2.9716$    	& $0.4276$  & $3.0542$    & $0.5129$         \\
    GMl-ASLCS$_{!Ms}$    	& $0.4036$ 		& $0.1209$ 		& $1.9306$ 		& $0.1653$  & $1.8301$    & $0.2787$             \\
    GMl-ASLCS$_{!Ma}$    	& $0.5393$    	& $0.1295$    	& $1.9516$    & $0.17659$ & $1.5916$ & $0.2470$  
\\ \hline \\ [-2ex]
    GMl-ASLCS$_{f}$      	& $1.7515$    & $0.5357$    & $3.0587$    & $\textbf{0.4620}$  & $\textbf{3.7674}$    & $0.5141$           \\
    GMl-ASLCS$_{d}$      	& $1.7746$    & $\textbf{1.0167}$    & $\textbf{3.2365}$    & $0.4425$  & $2.9760$    & $0.4945$             \\
    GMl-ASLCS$_{!\#}$    	& $1.7297$    & $0.5080$    & $3.0827$    & $0.4140$  & $3.0057$    & $0.5041$            \\
    GMl-ASLCS$_{Cl.sp}$  	& $\textbf{1.9291}$    & $0.5083$    & $3.0375$    & $0.4351$		& $2.9339$    & $0.4977$   
\\
    GMl-ASLCS$_{Cl.lsp}$ 	& $1.8003$    & $0.4648$    & $3.0854$    & $0.3955$  & $3.1086$    & $0.5098$             \\
    GMl-ASLCS$_{Cl.lge}$ 	& $1.6784$    & $0.4546$    & $3.10178$   & $0.4459$  & $3.3272$    & $0.4983$    
\\ \hline
    \end{tabular}
    \end{center}
\end{table}


\section{Σύνοψη}
Στο παρόν κεφάλαιο αξιολογήσαμε την ικανότητα ταξινόμησης του GMl-ASLCS σε έξι διαδεδομένα σύνολα πολυκατηγορικών δεδομένων, τα music, yeast, genbase, scene, medical και enron. Αρχικά, συγκρίναμε τις επιδόσεις του σε σχέση με τον GMl-ASLCS$_{\:0}$ και βρήκαμε πως για τη μετρική της ακρίβειας, ο GMl-ASLCS εμφανίζει στατιστικά σημαντική διαφορά σε σχέση με τον προκάτοχό του, βελτιώνοντας τις τιμές της παραπάνω μετρικής σε όλα τα πραγματικά σύνολα δεδομένων που εξετάσαμε (Παρ. \ref{subsec:accBasedComparison}). Αντίθετα, για τη μετρική της ακριβούς ορθότητας, οι GMl-ASLCS και GMl-ASLCS$_{\:0}$ δεν εμφανίζουν στατιστικά σημαντικές διαφορές (Παρ. \ref{subsec:exBasedComparison}). 

Στη συνέχεια, συγκρίναμε τις επιδόσεις του GMl-ASLCS με αυτές των διαδεδομένων στη βιβλιογραφία αλγορίθμων πολυκατηγορικής ταξινόμησης, οι οποίοι μάλιστα δεν ανήκουν στην οικογένεια των ΜαΣΤ. Βρήκαμε πως αν και ο GMl-ASLCS κατατάσσεται πρώτος ανάμεσά τους για τη μετρική της ακρίβειας και δεύτερος για αυτή της ακριβούς ορθότητας, συνολικά, δεν διακρίνεται κάποια στατιστικά σημαντική διαφορά στις επιδόσεις τους για τις δύο αυτές μετρικές (Παρ. \ref{subsec:accBasedComparison} και \ref{subsec:exBasedComparison}).

Όσον αφορά στην επίδραση του τροποποιημένου τελεστή διασταύρωσης, τα αποτελέσματα δείχνουν πως η πρωτύτερη υλοποίηση, η Διασταύρωση Ενός Σημείου, εμφανίζει χειρότερη συμπεριφορά από την υλοποίηση που επινοήθηκε ώστε να προσιδιάζει στη φύση της πολυκατηγορικής ταξινόμησης, τη Διασταύρωση Δύο Τμημάτων, σε κάθε πολυκατηγορικό σύνολο δεδομένων που χρησιμοποιήθηκε. Η βελτιωμένη συμπεριφορά του νέου τελεστή διασταύρωσης αφορά στις μετρικές της ακρίβειας (Πίνακας \ref{table:accuracyBasedComparisonVars}), της ακριβούς ορθότητας (Πίνακας \ref{table:exactMatchBasedComparisonVars}), αλλά και της μέσης κάλυψης δειγμάτων (Πίνακας \ref{table:coverageBasedComparisonVars}).

Διαπιστώσαμε, επιπλέον, πως ο GMl-ASLCS βελτιώνει συνολικά, όχι μόνο την προβλεπτική του ικανότητα, αλλά και την ικανότητά του να γενικεύει με ακρίβεια πάνω στα δείγματα των συνόλων δεδομένων, σε μεγαλύτερο βαθμό από ότι ο GMl-ASLCS$_{\:0}$, λόγω της λειτουργίας διαγραφής κανόνων από τα Match Sets που επινοήσαμε (Πίνακες \ref{table:gmlaslcs0coverage} και \ref{table:gmlaslcsCoverage}).

Παρατηρήσαμε πως η λειτουργία διαγραφής κανόνων από τα Match Sets είναι ένα αναντικατάστατο κομμάτι του αλγορίθμου GMl-ASLCS, όχι μόνο αυξάνοντας τον αριθμό δειγμάτων που καλύπτουν κατά μέσο όρο οι κανόνες ενός ΜαΣΤ (πιν. \ref{table:coverageBasedComparisonVars}), όπως αναφέραμε παραπάνω, αλλά βελτιώνοντας τη συνολική προβλεπτική ικανότητα του μοντέλου που κατασκευάζει ο GMl-ASLCS σε κάθε πολυκατηγορικό πρόβλημα. 

Στους Πίνακες \ref{table:gmlaslcs0coverage} και \ref{table:gmlaslcsCoverage} φαίνεται η σημαντική βελτίωση που επέφερε η απαγόρευση συμμετοχής κανόνων μηδενικής κάλυψης στον πληθυσμό ενός ΜαΣΤ, καθώς πλέον η εξελικτική διαδικασία καθίσταται σε μεγαλύτερο βαθμό ελέγξιμη και αδιάλειπτη και, συνεπώς, η εργασία του χρήστη ή/και ερευνητή τού επιτρέπει να εξάγει ακριβέστερα συμπεράσματα για την εν μέρει και εν συνόλω συμπεριφορά ενός ΜαΣΤ.

Τέλος, εξάγαμε ενδιαφέροντα συμπεράσματα συγκρίνοντας τον GMl-ASLCS με τις διάφορες τροποποιήσεις που εισάγαμε στην Εν. \ref{sec:alterations}. Οι αλγόριθμοι που βασίζονται στον GMl-ASLCS και περιλαμβάνουν α) τη μεταβολή των παραμέτρων $\omega, \phi$ από $(\omega, \phi) \equiv (0.9, 1)$ σε $(\omega, \phi) \equiv (0, 0)$ και β) την αρχικοποίηση του πληθυσμού ενός ΜαΣΤ μέσω της ομαδοποίησης, με $\gamma=0.2$ και μηδενικές πιθανότητες γενίκευσης των γνωρισμάτων και ετικετών των κεντροειδών, αποδεικνύονται συνεπείς και, εν γένει, εύρωστοι (Πίνακας \ref{table:accuracyBasedComparisonVars} και \ref{table:accuracyBasedComparisonVars}). Αν και οι επιδόσεις τους υπολείπονται αυτών του GMl-ASLCS για τα έξι πραγματικά σύνολα δεδομένων, περαιτέρω  πιο στοχευμένα πειράματα ίσως φέρουν κάποιον από τους δύο παραπάνω αλγορίθμους σε μεγαλύτερα επίπεδα ακρίβειας από αυτόν.




